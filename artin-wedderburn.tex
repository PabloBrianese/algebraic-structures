\documentclass{report}
\usepackage[spanish]{babel}

\usepackage{mathtools}
\usepackage{amsthm, amsfonts}

\newcommand{\naturalNumbers}{\mathbb{N}}
\newcommand{\Hom}{\text{Hom}}

\DeclareMathOperator{\image}{\text{Im}}

\newtheorem{theorem}{Teorema}
\newtheorem{lemma}{Lema}
\newtheorem{proposition}{Proposición}
\newtheorem{definition}{Definición}

\title{Teorema de Artin--Wedderburn}
\author{Pablo Brianese}
\begin{document}
  \maketitle

  \begin{theorem}[Lema de Zorn]
    Si \(A\) es un conjunto parcialmente ordenado novacío tal que toda cadena en \(A\) tiene una cota superior en \(A\), entonces \(A\) contiene un elemento maximal.
  \end{theorem}

  \begin{definition}
    Un módulo (izquierdo) \(A\) sobre un anillo \(R\) es \emph{simple} (o \emph{irreducible}) si \(R A \neq 0\) y \(A\) no tiene submódulos propios.
    Un anillo \(R\) es \emph{simple} si \(R^2 \neq 0\) y \(R\) no tiene ideales (bilaterales) propios.
  \end{definition}

  \begin{proposition}
    Todo módulo simple \(A\) es cíclico; de hecho, \(A = R a\) para todo \(a \in A\) nonulo.
  \end{proposition}
  \begin{proof}
    Ambos \(Ra\) (con \(a \in A\) nonulo) y \(B = \{c \in A : R c = 0\}\) son submódulos de \(A\), de aquí que por simplicidad cada uno de ellos sea igual a 0 o \(A\).
    También por simplicidad \(R A \neq 0\), esto implica \(B \neq A\) y \(B = 0\).
    Luego \(a \notin B\) y \(R a \neq 0\).
    En conclusión \(R a = A\).
  \end{proof}

  \begin{definition}
    Un módulo (izquierdo) \(A\) es \emph{fiel} si su aniquilador (izquierdo) \(\mathcal{A}(A)\) es 0.
    Un anillo \(R\) es \emph{primitivo} (\emph{izquierdo}) si existe un \(R\)-módulo simple y fiel.
  \end{definition}

  Los anillos primitivos derechos se definen análogamente.
  Sí existen anillos primitivos derechos que no son primitivos izquierdos.
  De aquí en más \emph{primitivo} siempre significará \emph{primitivo izquierdo}.
  Sin embargo, todos los resultados probados para anillos primitivos izquierdos son verdaderos, mutatis mutandis, para anillos primitivos derechos.

  \begin{definition}
    Sea \(V\) un espacio vectorial izquierdo sobre un anillo de división \(D\).
    Un subanillo \(R\) del anillo de endomorfismos \(\Hom_D(V, V)\) es un \emph{anillo denso de endomorfismos} de \(V\) (o un \emph{subanillo denso} de \(\Hom_D(V, V)\)) si para todo entero positivo \(n\), cada subconjunto linealmente independiente \(\{u_1, \dots, u_n\}\) de \(V\) y cada subconjunto arbitrario \(\{v_1, \dots, v_n\}\) de \(V\), existe \(\theta \in R\) tal que \(\theta(u_i) = v_i\) \((\forall i \in \{1, \dots, n\})\).
  \end{definition}

  \begin{lemma}
    Sea \(A\) un módulo simple sobre un anillo \(R\).
    Consideramos \(A\) como un espacio vectorial sobre el anillo de división \(D = \Hom_R(A, A)\).
    Si \(V\) es un subespacio finito-dimensional del \(D\)-espacio vectorial \(A\) y \(a \in A \setminus V\), entonces existe \(r \in R\) tal que \(r a \neq 0\) y \(r V = 0\).
  \end{lemma}
  \begin{proof}
    La prueba es por inducción sobre \(n = \dim_D V\).
    Comenzamos por el caso base.
    Si \(n = 0\), entonces \(V = 0\) y \(a \neq 0\).
    Porque \(A\) es simple, \(a \neq 0\) implica \(R a = A\).
    Consecuentemente existe \(r \in R\) tal que \(r a = a \neq 0\) y \(r V = r 0 = 0\).

    En el paso inductivo, supongamos \(\dim_D V = n > 0\) y que el teorema es verdadero para dimensiones menores a \(n\).
    Sea \(\{u_1, \dots, u_{n - 1}, u\}\) una \(D\)-base de \(V\) y sea \(W\) el subespacio \((n - 1)\)-dimensional generado por \(\{u_1, \dots, u_{n - 1}\}\) (siendo \(W = 0\) cuando \(n = 1\)).
    Entonces \(V = W \oplus D u\) (suma directa de espacios vectoriales).
    Nuestra hipótesis inductiva tiene dos consecuencias importantes:
    \begin{enumerate}
      \item para todo \(v \in A \setminus W\) existe \(r \in R\) tal que \(r u \neq 0\) y \(r W = 0\);
      \item para todo \(v \in A\), si \(r v = 0\) para todo \(r \in R\) entonces \(v \in W\).
    \end{enumerate}
    La primera consecuencia implica que existe \(r \in R\) tal que \(r u \neq 0\) y \(r W = 0\).
    Pero \(r W = 0\) si y solo si \(r \in \mathcal{A}(W)\), siendo \(I = \mathcal{A}(W)\) un ideal izquierdo de \(R\).
    Además \(r u \in I u \setminus 0\), siendo \(I u\) un submódulo de \(A\).
    Por simplicidad, este submódulo nonulo debe ser \(I u = A\).

    Para terminar el argumento inductivo, debemos encontrar \(r \in R\) tal que \(r a \neq 0\) y \(r V = 0\).
    Si no existe tal \(r\), entonces podemos definir una aplicación \(\theta : A \rightarrow A\) como sigue.
    Para \(r u \in I u = A\) definimos \(\theta(r u) = r a \in A\).
    Afirmamos que \(\theta\) está bien definida.
    Sean \(r_1, r_2 \in I\) tales que \(r_1 u = r_2 u\).
    Por hipótesis \((r_1 - r_2) a = 0\) o \((r_1 - r_2) V \neq 0\).
    Ahora bien, porque \(r_1 - r_2 \in I = \mathcal{A}(W)\) tenemos \((r_1 - r_2) W = 0\);
    y porque \(D = \Hom_D(A, A)\), para cada \(d \in D\) tenemos \((r_1 - r_2) (d \cdot u) = (r_1 - r_2) d(u) = d((r_1 - r_2) u) = d(0) = 0\).
    Juntos, estos dos datos implican \((r_1 - r_2) V = (r_1 - r_2) (W \oplus D u) = 0\).
    Consecuentemente, por hipótesis \((r_1 - r_2) a = 0\).
    Por lo tanto \(\theta(r_1 u) = r_1 a = r_2 a = \theta(r_2 u)\).
    Podemos mostrar que \(\theta \in \Hom_D(A, A) = D\).
    Luego para cada \(r \in I\), \(0 = \theta(r u) - r a = r \theta(u) - r a = r(\theta(u) - a)\).
    De aquí que \(\theta(u) - a \in W\), por la segunda consecuencia de la hipótesis inductiva.
    Consecuentemente \(a = \theta u - (\theta u - a) \in D u + W = V\),
    lo cual contradice el hecho \(a \notin V\).
    Por lo tanto, existe \(r \in R\) tal que \(r a \neq 0\) y \(r V = 0\).
  \end{proof}

  \begin{theorem}[de Densidad de Jacobson]
    Sea \(R\) un anillo primitivo y \(A\) un \(R\)--módulo simple y fiel.
    Considerar \(A\) como espacio vectorial sobre el anillo de división \(\Hom_R(A, A) = D\).
    Entonces \(R\) es isomorfo a un anillo denso de endomorfismos de \(D\)--espacio vectorial \(A\).
  \end{theorem}
  \begin{proof}
    Para cada \(r \in R\) la aplicación \(\alpha_r : A \rightarrow A\) dada por \(\alpha_r(a) = r a\) es facilmente identificada como un \(D\)--endomorfismo de \(A\): esto es, \(\alpha_r \in \Hom_D(A, A)\).
    Además para todo par \(r, s \in R\) se verifican \(\alpha_{(r + s)} = \alpha_r + \alpha_s\) y \(\alpha_{r s} = \alpha_r \alpha_s\).
    Consecuentemente la aplicación \(\alpha : R \rightarrow \Hom_D(A, A)\) definida por \(\alpha(r) = \alpha_r\) es un homomorfismo de anillos bien definido.
    Dado que \(A\) es un \(R\)--módulo fiel, \(\alpha_r = 0\) si y solo si \(r \in \mathcal{A}(A) = 0\).
    De aquí que \(\alpha\) es un monomorfismo, y \(R\) es isomorfo al subanillo \(\image \alpha\) de \(\Hom_D(A, A)\).

    Para completar la prueba debemos mostrar que \(\image \alpha\) es un subanillo denso de \(\Hom_D(A, A)\).
    Dado un subconjunto \(D\)--linealmente independiente \(\{u_1, \dots, u_n\}\) de \(A\), y un subconjunto arbitrario \(\{v_1, \dots, v_n\}\) de \(A\), debemos encontrar \(\alpha_r \in \image \alpha\) tal que \(\alpha_r(u_i) = v_i\) \((\forall i \in \{1, \dots, n\})\).
    Para cada \(i\) sea \(V_i\) el \(D\)--subespacio de \(A\) generado por \(\{u_j : j \neq i\}\).
    Dado que \(\{u_1, \dots, u_n\}\) es linealmente independiente, \(u_i \notin V_i\).
    Consecuentemente, por el lema \ref{} existe \(r_i \in R\) tal que \(r_i u_i \neq 0\) y \(r_i V_i = 0\).
    Después aplicamos el lema \ref{} al subespacio nulo y a elemento nonulo \(r_i u_i\):
    existe \(s_i \in R\) tal que \(s_i r_i u_i \neq 0\) y \(s_i 0 = 0\).
    Siendo \(s_i r_i u_i \neq 0\), el \(R\) submódulo \(R (r_i u_i)\) de \(A\) es nonulo, luego \(R(r_i u_i) = A\) por simplicidad.
    Por esto existe \(t_i \in R\) tal que \(t_i r_i u_i = v_i\).
    Sea \(r = t_1 r_1 + t_2 r_2 + \cdots + t_n r_n\).
    Recordar que \(u_i \in V_j\) para \(i \neq j\), luego \(t_j r_j u_i \in t_j (r_j V_i) = t_j 0 = 0\).
    Consecuentemente \(\alpha_r(u_i) = (t_1 r_1 + \cdots + t_n r_n) u_i = r_i r_i u_i = v_i\).
    Por lo tanto \(\image \alpha\) es un anillo denso de endomorfismos de \(D\)--espacio vectorial \(A\).
  \end{proof}

  \begin{definition}
    Decimos que un módulo \(A\) satisface la \emph{condición de la cadena ascendente (ACC) sobre submódulos} (o decimos que es \emph{noetheriano}) si para toda cadena \(A_1 \subseteq A_2 \subseteq A_3 \subseteq \cdots\) de submódulos de \(A\), existe un entero \(m\) tal que \(B_i = B_m\) para todo \(i \geq m\).
  \end{definition}

  Si un anillo \(R\) es pensado como módulo izquierdo (resp. derecho) sobre si mismo, entonces es facil ver que los submódulos de \(R\) son precisamente los ideales izquierdos (resp. derechos) de \(R\).
  Consecuentemente, en este caso se acostumbra hablar de condiciones de cadena sobre ideales (izquierdos o derechos) en lugar de submódulos.

  \begin{definition}
    Un anillo \(R\) es \emph{noetheriano izquierdo} (resp. \emph{derecho}) si \(R\) satisface la condición de la cadena ascendente sobre ss ideales izquierdos (resp. derechos).
    Se dice que \(R\) es \emph{noetheriano} si \(R\) es noetheriano izquierdo y derecho a la vez.

    Un anillo \(R\) es \emph{artiniano izquierdo} (resp. \emph{derecho}) si \(R\) satisface la condición de la cadena descendiente sobre sus ideales izquierdos (resp. derechos).
    Se dice que \(R\) es artiniano si \(R\) es artiniano izquierdo y derecho a la vez.
  \end{definition}

  \begin{theorem}[de Artin--Wedderburn]
    Las siguientes condiciones sobre un anillo artiniano izquierdo \(R\) son equivalentes.
    \begin{enumerate}
      \item
        \label{property:artin--wedderburn_simple-ring}
        \(R\) es simple;
      \item
        \label{property:artin--wedderburn_primitive-ring}
        \(R\) es primitivo;
      \item
        \label{property:artin--wedderburn_endomorphism-ring}
        \(R\) es isomorfo al anillo de endomorfismos de un espacio vectorial nonulo sobre un anillo de división \(D\);
      \item
        \label{property:artin--wedderburn_matrix-ring}
        para algún entero positivo \(n\), \(R\) es isomorfo al anillo formado por las matrices \(n \times n\) sobre un anillo de división.
    \end{enumerate}
  \end{theorem}
  \begin{proof}
    \(\ref{property:artin--wedderburn_simple-ring} \Rightarrow \ref{property:artin--wedderburn_primitive-ring}\).
    Primero observamos que \(I = \{r \in R \mid R r = 0\}\) es un ideal de \(R\), con la propiedad \(I R = 0\).
    Pero \(R\) es simple: no tiene ideales propios, por lo cual \(I = R\) o \(I = 0\);
    y \(R R \neq 0\), por lo cual \(I = 0\).

    Consideremos el conjunto \(\mathcal{S}\) formado por todos los ideales izquierdos nonulos de \(R\).
    Dado que \(R\) es artiniano izquierdo, satisface la condición de la cadena descendiente sobre ideales izquierdos.
    En particular, para toda sucesión \(\{S_i\}_{i \in \naturalNumbers}\) en \(\mathcal{S}\) con \(S_0 \supseteq S_1 \supseteq S_2 \supseteq \cdots\), existe un \(m \in \naturalNumbers\) tal que \(S_m = S_i\) para todo \(i \geq m\).
    El Lema de Zorn permite deducir de esto la existencia de un elemento minimal \(J \in \mathcal{S}\), tal que \(J \supseteq J' \rightarrow J = J'\) para todo \(J' \in \mathcal{S}\).
    Esta minimalidad hace que \(J\) no tenga \(R\)--submódulos propios (un \(R\)--submódulo de \(J\) es un ideal izquierdo de \(R\) contenido en \(J\)).

    Afirmamos que el aniquilador izquierdo \(\mathcal{A}(J)\) de \(J\) en \(R\) es cero.
    De otro modo \(\mathcal{A}(J) = R\) por simplicidad y \(R u = 0\) para cada \(u \in J\) nonulo.
    Consecuentemente, cada uno de estos \(u\) nonulos pertenece a \(I = 0\), lo cual es una contradicción.
    Por lo tanto \(\mathcal{A}(J) = 0\) y \(R J \neq 0\).
    En conclusión, \(J\) es un \(R\)--módulo simple y fiel, y \(R\) es primitivo.

    \(\ref{property:artin--wedderburn_primitive-ring} \Rightarrow \ref{property:artin--wedderburn_endomorphism-ring}\)
    Por el Teorema de Densidad de Jacobson \ref{}, \(R\) es isomorfo a un anillo denso \(T\) compuesto por endomorfismos de un espacio vectorial \(V\) sobre un anillo de división \(D\).
    Porque \(R\) es artiniano izquierdo, \(R \simeq T = \Hom _D(V, V)\) por el teorema \ref{}.

    \(\ref{property:artin--wedderburn_endomorphism-ring} \Leftrightarrow \ref{property:artin--wedderburn_matrix-ring}\)
    Teorema \ref{}

    \(\ref{property:artin--wedderburn_matrix-ring} \Leftrightarrow \ref{property:artin--wedderburn_simple-ring}\)
    Ejercicio \ref{}
  \end{proof}









































\end{document}