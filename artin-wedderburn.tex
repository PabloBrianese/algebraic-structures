\documentclass{report}
\usepackage[spanish]{babel}

\usepackage{mathtools}
\usepackage{amsthm}

\newtheorem{theorem}{Teorema}
\newtheorem{definition}{Definición}

\title{Teorema de Artin--Wedderburn}
\author{Pablo Brianese}
\begin{document}
  \maketitle

  \begin{definition}
    Un anillo \(R\) es \emph{noetheriano izquierdo} (resp. \emph{derecho}) si \(R\) satisface la condición de la cadena ascendente sobre ss ideales izquierdos (resp. derechos).
    Se dice que \(R\) es \emph{noetheriano} si \(R\) es noetheriano izquierdo y derecho a la vez.

    Un anillo \(R\) es \emph{artiniano izquierdo} (resp. \emph{derecho}) si \(R\) satisface la condición de la cadena descendiente sobre sus ideales izquierdos (resp. derechos).
    Se dice que \(R\) es artiniano si \(R\) es artiniano izquierdo y derecho a la vez.
  \end{definition}

  \begin{theorem}[de Artin--Wedderburn]
    Las siguientes condiciones sobre un anillo artiniano izquierdo \(R\) son equivalentes.
    \begin{enumerate}
      \item \(R\) es simple;
      \item \(R\) es isomorfo al anillo de endomorfismos de un espacio vectorial nonulo sobre un anillo de división \(D\);
      \item para algún entero positivo \(n\), \(R\) es isomorfo al anillo formado por las matrices \(n \times n\) sobre un anillo de división.
    \end{enumerate}
  \end{theorem}

\end{document}