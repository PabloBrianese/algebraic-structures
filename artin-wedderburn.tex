\documentclass{report}
\usepackage[spanish]{babel}
\usepackage{hyperref}

\usepackage{mathtools}
\usepackage{amsthm, amsfonts}

\newcommand{\naturalNumbers}{\mathbb{N}}
\newcommand{\Hom}{\textnormal{Hom}}
\newcommand{\End}{\textnormal{End}}
\newcommand{\Mat}{\textnormal{Mat}}
\newcommand{\Col}{\textnormal{Col}}
\newcommand{\Fila}{\textnormal{Fila}}

\DeclareMathOperator{\image}{\text{Im}}

\newtheorem{theorem}{Teorema}
\newtheorem{corollary}{Corolario}
\newtheorem{lemma}{Lema}
\newtheorem{proposition}{Proposición}
\newtheorem{definition}{Definición}
\newtheorem{remark}{Observación}

\title{Teorema de Artin--Wedderburn}
\author{Pablo Brianese}
\begin{document}
  \maketitle
  \tableofcontents
  \chapter{Matrices}

  En el presente capítulo estudiaremos matrices.
  Matrices cuadradas con entradas en un anillo de división.
  Nuestro objetivo final será probar que el anillo formado por estas es artiniano y noetheriano.
  Ambas nociones refieren a cadenas de ideales del anillo, para conocerlas usaremos las llamadas series de composición.
  Por un lado, trabajaremos con series de composición concretas.
  Por el otro, estudiaremos la relación abstracta que existe entre las series de composición y las cadenas de ideales.

  Sea \(D\) un anillo de división, escribimos \(\Mat_n D\) para denotar al anillo formado por las matrices \(n \times n\) con entradas en \(D\).
  Nuestra principal herramienta a la hora de calcular dentro del anillo \(\Mat_n D\) seran las matrices elementales \(E_{r s}\), donde \(r, s \in \{1, \dots, n\}\), y \(E_{r s}\) tiene \(1_D\) como entrada \((r, s)\) y \(0_D\) en las demás posiciones.
  Parte de su valor se explica en la siguiente proposición.
  \begin{proposition}
    \label{proposition:matrixEntryCalculation}
    Sea \(D\) un anillo de división y \(R = \Mat_n D\).
    Para toda matriz \(A = {(A_{i j})}_{i j}\) en \(R\)
    \begin{align}
      E_{p r} A E_{s q}
      =
      A_{r s} E_{p q}
    \end{align}
  \end{proposition}
  \begin{proof}
    Es un cálculo directo.
  \end{proof}

  Usamos esta proposición para entender algunos ideales cíclicos de \(\Mat_n D\), con propiedades especiales.

  \begin{theorem}\label{theorem:matricesWhereAlmostEveryColumnIsNull}
    Si \(D\) es un anillo de división y \(R = \Mat_n D\), entonces 
    \begin{align}
      R E_{j_0 j_0}
      =
      \sum_{i = 1}^n D E_{i j_0}
      &&(\forall j_0 \in \{1, \dots, n\})
    \end{align}
  \end{theorem}
  \begin{proof}
    Fijamos \(j_0 \in \{1, \dots, n\}\), y escribimos \(I = R E_{j_0 j_0}\).

    Afirmamos que \(I' = \sum_{i = 1}^n D E_{i j_0}\) es igual a \(I\).
    Lo demostraremos usando que, por la proposición \ref{proposition:matrixEntryCalculation}, toda matriz \(a = {(a_{i j})}_{i j}\) en \(R\) verifica
    \begin{align}
      a E_{j_0 j_0}
      =
      I_n a E_{j_0 j_0}
      =
      \sum_{i = 1}^n E_{i i} a E_{j_0 j_0}
      =
      \sum_{i = 1}^n a_{i j_0} E_{i j_0}
    \end{align}
    Si \(A \in I\), entonces existe \(a \in R\) con \(A = a E_{j_0 j_0}\).
    Luego \(A = a E_{j_0 j_0} = \sum_{i = 1}^n a_{i j_0} E_{i j_0}\), también pertenece a \(I'\).
    Recíprocamente, si \(A = (A_{i j})_{i j}\) pertenece \(I'\), entonces \(A = \sum_{i = 1}^n A_{i j_0} E_{i j_0} = A E_{j_0 j_0}\), y también es un elemento de \(I\).
  \end{proof}

  Argumentos análogos demuestran que
  \begin{theorem}
    \label{theorem:matricesWhereAlmostEveryRowIsNull}
    Si \(D\) es un anillo de división y \(R = \Mat_n D\), entonces
    \begin{align}
      \displaystyle
      E_{i_0 i_0} R
      =
      \sum_{j = 1}^n E_{i_0 j} D
      &&(\forall i_0 \in \{1, \dots, n\})
    \end{align}
  \end{theorem}
  \begin{proof}
    Fijemos \(i_0 \in \{1, \dots, n\}\), y escribamos \(I = E_{i_0 i_0} R\).

    Afirmamos que \(I' = \sum_{j = 1}^n E_{i_0 j} D\) es igual a \(I\).
    Lo demostraremos usando que, por la proposición \ref{proposition:matrixEntryCalculation}, toda matriz \(a = (a_{i j})_{i j}\) en \(R\) verifica
    \begin{align}
      E_{i_0 i_0} a
      =
      E_{i_0 i_0} a I_n
      =
      \sum_{j = 1}^n E_{i_0 i_0} a E_{j j}
      =
      \sum_{j = 1}^n a_{i_0 j} E_{i_0 j}
    \end{align}
    Si \(A \in I\) entonces existe \(a \in R\) con \(A = E_{i_0 i_0} a\).
    Luego \(A = E_{i_0 i_0} a = \sum_{j = 1}^n a_{i_0 j} E_{i_0 j}\), también pertenece a \(I'\).
    Recíprocamente, si \(A = (A_{i j})_{i j}\) pertenece a \(I'\), entonces \(A = \sum_{j = 1}^n A_{i_0 j} E_{i_0 j} = E_{i_0 i_0} A\), y también es un elemento de \(I\).
  \end{proof}

  Ahora probaremos propiedades importantes de estos módulos cíclicos.

  \begin{theorem}
    Si \(D\) es un anillo de división y \(R = \Mat_n D\),
    entonces los \(R\)-submódulos izquierdos de \(R\)
    \begin{align}
      R E_{j j}
      &&(j \in \{1, \dots, n\})
    \end{align}
    no tienen submódulos propios.
  \end{theorem}
  \begin{proof}
    Fijemos \(j_0 \in \{1, \dots, n\}\), y escribamos \(E = E_{j_0 j_0}\), \(I = R E\).

    Supongamos que \(J\) es un submódulo nonulo de \(I\).
    Probaremos \(J = I\).
    Por hipótesis existe \(a \in J \setminus 0\).
    Porque \(a \in I\), se sigue del teorema \ref{theorem:matricesWhereAlmostEveryColumnIsNull} que \(a = \sum_{i = 1}^n a_{i j_0} E_{i j_0}\).
    Notemos que \(a \neq 0\) implica \(a_{i_0 j_0} \neq 0\) para un \(i_0 \in \{1, \dots, n\}\).
    Porque \(D\) es un anillo de división, existe una matriz elemental de transformación \(M\), que actúa sobre \(a\) multiplicando (por izquierda) su fila \(i_0\) por el elemento \(a_{i_0 j_0}^{- 1} \in D\).
    Entonces \(M a = E_{i_0 j_0} + \sum_{i \neq i_0} a_{i j_0} E_{i j_0}\).
    Luego, existen matrices elementales de transformación \(A_i\) \((i \in \{1, \dots, n\} \setminus i_0)\), que actúan sobre \(a\) sumando a la fila \(i\)-ésima el producto (por izquierda) de \(- a_{i j_0}\) con la fila \(i_0\)-ésima.
    Entonces \(A_1 \cdots A_n M a = E_{i_0 j_0}\) (donde definimos \(A_{i_0} = I_n\), la matriz identidad, para mejorar la notación).
    Finalmente, para cada \(i \in \{1, \dots, n\}\), existe una matriz elemental de transformación \(P_i\) que actúa sobre \(a\) permutando las filas \(i\) e \(i_0\).
    De ese modo \(P_i A_1 \cdots A_n M a = E_{i j_0}\).
    Por lo tanto \(E_{i j_0} \in J\) para todo \(i \in \{1, \dots, n\}\).
    Eso implica que \(I = \sum_{i = 1}^n D E_{i j_0} \subseteq J\) (recordar el teorema \ref{theorem:matricesWhereAlmostEveryColumnIsNull}).
    En conclusión \(J = I\).
  \end{proof}

  Argumentos similares demuestran que
  \begin{theorem}
    Si \(D\) es un anillo de división y \(R = \Mat_n D\), entonces los \(R\)-submódulos derechos de \(R\)
    \begin{align}
      E_{i i} R
      &&(i \in \{1, \dots, n\})
    \end{align}
    no tienen submódulos propios.
  \end{theorem}
  \begin{proof}
    Fijemos \(i_0 \in \{1, \dots, n\}\), y escribamos \(E = E_{i_0 i_0}\), \(I = E R\).

    Supongamos que \(J\) es un submódulo nonulo de \(I\).
    Entonces existe \(a \in J \setminus 0\).
    Porque \(a \in I\) se sigue del teorema \ref{theorem:matricesWhereAlmostEveryRowIsNull} que \(a = \sum_{j = 1}^n a_{i_0 j} E_{i_0 j}\).
    Notemos que \(a \neq 0\) implica \(a_{i_0 j_0} \neq 0\) para un \(j_0 \in \{1, \dots, n\}\).
    Porque \(D\) es un anillo de división, existe una matriz elemental de transformación \(M\), que actúa sobre \(a\) multiplicando (por derecha) su columna \(j_0\) por el elemento \(a_{i_0 j_0}^{- 1} \in D\).
    Entonces \(a M = E_{i_0 j_0} 1_D + \sum_{j \neq j_0} a_{i_0 j} E_{i_0 j}\).
    Luego, existen matrices elementales de transformación \(A_j\) \((j \in \{1, \dots, n\} \setminus j_0)\), que actúan sobre \(a\) sumando a la columna \(j\)-ésima el producto (por derecha) de \(- a_{i_0 j}\) con la columna \(j_0\)-ésima.
    Entonces \(a M A_1 \cdots A_n = E_{i_0 j_0} 1_D\) (donde definimos \(A_{j_0} = I_n\), la matriz identidad, para mejorar la notación).
    Finalmente, para cada \(j \in \{1, \dots, n\}\), existe una matriz elemental de transformación \(P_j\) que actúa sobre \(a\) permutando las columnas \(j\) y \(j_0\).
    De ese modo \(a M A_1 \cdots A_n P_j = E_{i_0 j} 1_D\).
    Por lo tanto \(E_{i_0 j} 1_D \in J\) para todo \(j \in \{1, \dots, n\}\).
    Eso implica que \(I = \sum_{j = 1}^n E_{i_0 j} D \subseteq J\) (recordar el teorema \ref{theorem:matricesWhereAlmostEveryRowIsNull}).
    En conclusión \(J = I\).
  \end{proof}

  Estos módulos cíclicos que hemos estudiado aparecen como cocientes en cadenas formadas por otros módulos cíclicos, que pasamos a investigar.

  \begin{theorem}
    \label{theorem:compositionSeriesOfLeftModulesForMatrixRing}
    Sea \(D\) un anillo de división y \(R = \Mat_n D\).
    Definimos \(M_0 = 0\) y para \(j \in \{1, \dots, n\}\) definimos \(M_j = R (E_{1 1} + \cdots + E_{j j})\).
    Entonces la secuencia \(R = M_n \supseteq M_{n - 1} \supseteq \cdots \supseteq M_1 \supseteq M_0 = 0\), formada por \(R\)-submódulos izquierdos de \(R\), satisface \(M_j / M_{j - 1} \simeq R E_{j j}\).
  \end{theorem}
  \begin{proof}
    Primera observación.
    Sea \(j \in \{1, \dots, n\}\) arbitrario.
    Notar que \(M_j \subseteq R E_{1 1} + \cdots + R E_{j j}\).
    Usando el teorema \ref{theorem:matricesWhereAlmostEveryColumnIsNull}, se sigue que, para toda \(A \in M_j\), son nulas las columnas \(j + 1, \dots, n\) de \(A\).

    Fijemos \(j_0 \in \{1, \dots, n\}\), y simplifiquemos la notación escribiendo \(E = E_{j_0 j_0}\), \(M = M_{j_0}\), \(N = M_{j_0 - 1}\).

    Segunda observación.
    Notar que si \(A \in M\) con \(A = r (E_{1 1} + \cdots + E_{j_0 j_0})\) para un \(r \in R\), entonces \(A E = r E\).
    En efecto
    \begin{align}
      A E
      &=
      r (E_{11} + \cdots + E_{j_0 j_0}) E
      \\
      &=
      r (E_{11} E + \cdots + E_{j_0 - 1, j_0 - 1} E + E^2)
      \\
      &=
      r (0 + \cdots + 0 + E)
      \\
      &=
      r E
    \end{align}
    Por este motivo \(A + N = A E + N\).
    Para verlo, calculamos
    \begin{align}
      A + N
      &=
      r (E_{1 1} + \cdots + E_{j_0 j_0}) + N
      \\
      &=
      r(E_{1 1} + \cdots + E_{j_0 - 1, j_0 - 1}) + r E + N
      \\
      &=
      r E + N
      \\
      &=
      A E + N
    \end{align}

    Tercera observación.
    Supongamos que \(A + N = B + N\) para unas matrices \(A, B \in M\) arbitrarias.
    Entonces \(A E + N = B E + N\) por nuestra segunda observación.
    Escribamos \(C = (A - B) E\).
    Por un lado, la ecuación \(A E + N = B E + N\) implica \(C \in N\), y por el otro \(C \in R E\).
    El primer dato, mediante nuestra primera observación, implica \(\Col_j C = 0\) para \(j \in \{j_0, \dots, n\}\).
    El segundo dato, mediante el teorema \ref{theorem:matricesWhereAlmostEveryColumnIsNull}, implica \(\Col_j C = 0\) para \(j \in \{1, \dots, n\} \setminus j_0\).
    Luego \(C = 0\).
    Es decir \(A E = B E\).

    Esta última observación nos permite definir una función \(\phi : M / N \rightarrow R E\) dada por \(A + N \mapsto A E\).
    Resulta ser un homomorfismo de \(R\)-módulos.
    Es además un monomorfismo.
    Si \(\phi(A + N) = 0\) entonces \(A E = 0\) y \(\Col_{j_0} A = 0\).
    Además, \(A \in M\) implica \(\Col_j A = 0\) \((\forall j \in \{j_0 + 1, \dots, n\})\).
    Luego \(\Col_j A = 0\) \((\forall j \in \{j_0, \dots, n\})\), y \(A \in N\).
    Entonces \(A + N = 0 + N\), el elemento nulo del cociente.
    También es un epimorfismo.
    Dado \(a E \in R E\), tenemos \(a E \in M\) por el teorema \ref{theorem:matricesWhereAlmostEveryColumnIsNull}.
    Calculamos \(\phi(a E + N)= (a E) E = a E^2 = a E\).
    Por lo tanto \(a E \in \image \phi\).
    Concluímos que \(\phi : M / N \rightarrow R E\) es un isomorfismo.
  \end{proof}

  Argumentos análogos demuestran el siguiente teorema
  \begin{theorem}
    \label{theorem:compositionSeriesOfRightModulesForMatrixRing}
    Sea \(D\) un anillo de división y \(R = \Mat_n D\).
    Definimos \(M_0 = 0\) y para \(i \in \{1, \dots, n\}\) definimos \(M_i = (E_{1 1} + \cdots + E_{i i}) R\).
    Entonces la secuencia \(R = M_n \supseteq M_{n - 1} \supseteq \cdots \supseteq M_1 \supseteq M_0 = 0\), formada por \(R\)-módulos derechos, satisface \(M_i / M_{i - 1} \simeq E_{i i} R\).
  \end{theorem}
  \begin{proof}
    Primera observación.
    Sea \(i \in \{1, \dots, n\}\) arbitrario.
    Notar que \(M_i \subseteq E_{1 1} R + \cdots + E_{i i} R\).
    Usando el teorema \ref{theorem:matricesWhereAlmostEveryRowIsNull}, se sigue que, para toda \(A \in M_i\), son nulas las filas \(i + 1, \dots, n\) de \(A\).

    Fijemos \(i_0 \in \{1, \dots, n\}\), y simplifiquemos la notación escribiendo \(E = E_{i_0 i_0}\), \(M = M_{i_0}\), \(N = M_{i_0 - 1}\).

    Segunda observación.
    Notar que si \(A \in M\) con \(A = (E_{1 1} + \cdots + E_{i_0 i_0}) r\) para un \(r \in R\), entonces \(E A = E r\).
    En efecto
    \begin{align}
      E A
      &=
      E (E_{1 1} + \cdots + E_{i_0 i_0}) r
      \\
      &=
      (E E_{1 1} + \cdots + E E_{i_0 - 1, i_0 - 1} + E^2) r
      \\
      &=
      (0 + \cdots + 0 + E) r
      \\
      &=
      E r
    \end{align}
    Por este motivo \(A + N = A E + N\).
    Para verlo, calculamos
    \begin{align}
      A + N
      &=
      (E_{1 1} + \cdots + E_{i_0 i_0}) r + N
      \\
      &=
      (E_{1 1} + \cdots + E_{i_0 - 1, i_0 - 1}) r + E r + N
      \\
      &=
      E r + N
      \\
      &=
      E A + N
    \end{align}

    Tercera observación.
    Supongamos que \(A + N = B + M_{i_0 -1}\) para unas matrices \(A, B \in M\) arbitrarias.
    Entonces \(E A + N = E B + N\), por nuestra segunda observación.
    Escribamos \(C = E (A - B)\).
    Por un lado, la ecuación \(E A + N = E B + N\) implica \(C \in N\), y por el otro \(C \in E R\).
    El primer dato, mediante nuestra primera observación, implica \(\Fila_i C = 0\) para \(i \in \{i_0, \dots, n\}\).
    El segundo dato, mediante el teorema \ref{theorem:matricesWhereAlmostEveryRowIsNull}, implica \(\Fila_i C = 0\) para \(i \in \{1, \dots, n\} \setminus i_0\).
    Luego \(C = 0\).
    Es decir \(E A = E B\).

    Esta última observación nos permite definir una función \(\phi : M / N \rightarrow E R\) dada por \(A + N \mapsto E A\).
    Resulta ser un homomorfismo de \(R\)-módulos.
    Es además un monomorfismo.
    Si \(\phi(A + N) = 0\) entonces \(E A = 0\) y \(\Fila_{i_0} A = 0\).
    Además, \(A \in M\) implica \(\Fila_i A = 0\) \((\forall i \in \{i_0 + 1, \dots, n\})\).
    Luego \(\Fila_i A = 0\) \((\forall i \in \{i_0, \dots, n\})\), y \(A \in N\).
    Entonces \(A + N = 0 + N\), el elemento nulo del cociente.
    También es un epimorfismo.
    Dado \(E a \in E R\), tenemos \(E a \in M_i\) \ref{theorem:matricesWhereAlmostEveryRowIsNull}.
    Calculamos \(\phi(E a + N) = E (E a) = E^2 a = E a\).
    Por lo tanto \(E a \in \image \phi\).
    Concluímos que \(\phi : M / N \rightarrow E R\) es un isomorfismo.
  \end{proof}

  \section{Módulos}
  \begin{theorem}
    \label{theorem:submodulesOfQuotientModule}
    Si \(R\) es un anillo y \(B\) es un submódulo de un \(R\)-módulo \(A\), entonces existe una correspondencia uno-a-uno entre el conjunto de los submódulos de \(A\) que contienen a \(B\) y el conjunto de todos los submódulos de \(A / B\), dada por \(C \mapsto C / B\).
    Por tanto todo submódulo de \(A / B\) es de la forma \(C / B\), donde \(C\) es un submódulo de \(A\) que contiene a \(B\).
  \end{theorem}


  \section{Cadenas de ideales}

  \begin{definition}
    Decimos que un módulo \(A\) satisface la \emph{condición de la cadena ascendente sobre submódulos} (o decimos que es \emph{noetheriano}) si para toda cadena \(A_1 \subseteq A_2 \subseteq A_3 \subseteq \cdots\) de submódulos de \(A\), existe un entero \(m\) tal que \(B_i = B_m\) para todo \(i \geq m\).

    Decimos que un módulo \(B\) satisface la \emph{condición de la cadena descendente sobre submódulos} (o que es \emph{artiniano}) si para toda cadena \(B_1 \supseteq B_2 \supseteq B_3 \supseteq \cdots\) de submódulos de \(B\), existe un entero \(m\) tal que \(B_i = B_m\) para todo \(i \geq m\).
  \end{definition}

  Si un anillo \(R\) es pensado como módulo izquierdo (resp. derecho) sobre si mismo, entonces es facil ver que los submódulos de \(R\) son precisamente los ideales izquierdos (resp. derechos) de \(R\).
  Consecuentemente, en este caso se acostumbra hablar de condiciones de cadena sobre ideales (izquierdos o derechos) en lugar de submódulos.

  \begin{definition}
    Un anillo \(R\) es \emph{noetheriano izquierdo} (resp. \emph{derecho}) si \(R\) satisface la condición de la cadena ascendente sobre sus ideales izquierdos (resp. derechos).
    Se dice que \(R\) es \emph{noetheriano} si \(R\) es noetheriano izquierdo y derecho a la vez.

    Un anillo \(R\) es \emph{artiniano izquierdo} (resp. \emph{derecho}) si \(R\) satisface la condición de la cadena descendente sobre sus ideales izquierdos (resp. derechos).
    Se dice que \(R\) es \emph{artiniano} si \(R\) es artiniano izquierdo y derecho a la vez.
  \end{definition}

  \begin{definition}
    Un módulo \(A\) satisface la \emph{condición maximal} [resp. \emph{minimal}] \emph{sobre submódulos} si todo conjunto novacío de submódulos de \(A\) contiene un elemento maximal [resp. minimal] (con respecto al orden dado por la inclusión de conjuntos).
  \end{definition}

  \begin{theorem}
    \label{theorem:equivalenceOfChainConditions}
    Un módulo satisface la condición de la cadena ascendente [resp. descencendente] sobre submódulos si y solo si satisface la condición maximal [resp. minimal] sobre submódulos.
  \end{theorem}
  \begin{proof}
    Supongamos que el módulo \(A\) satisface la condición minimal sobre submódulos y que \(A_1 \supseteq A_2 \supseteq \cdots\) es una cadena de submódulos.
    Entonces el conjunto \(\{A_i \mid i \geq 1\}\) tiene un elemento minimal, digamos \(A_n\).
    Consecuentemente, para \(i \geq n\) tenemos \(A_n \supseteq A_i\) por hipótesis y \(A_n \subseteq A_i\) por minimalidad. Luego \(A_i = A_n\) para todo \(i \geq n\).
    Por lo tanto, \(A\) satisface la condición descendente de la cadena.

    Recíprocamente, supongamos que \(A\) satisface la condición de la cadena descendente, y \(S\) es un conjunto novacío de submódulos de \(A\).
    Para empezar, existe \(B_0 \in S\).
    Si \(S\) no tiene elemento minimal, entonces para todo submódulo \(B\) en \(S\) existe al menos un submódulo \(B'\) en \(S\) tal que \(B \supset B'\).
    Para cada \(B\) en \(S\), elegimos uno de estos \(B'\) (Axioma de Elección).
    Esta elección define una función \(f : S \rightarrow S\) mediante \(B \mapsto B'\).
    Por el Teorema de la Recursión, existe una función \(\phi : \naturalNumbers \rightarrow S\) tal que \(\phi(0) = B_0\) y \(\phi(n + 1) = f(\phi(n))\) \((\forall n \in \naturalNumbers)\).
    Por tanto si \(B_n = \phi(n)\) \((\forall n \in \naturalNumbers)\), entonces \(B_0 \supset B_1 \supset \cdots\) es una cadena descendente que viola la condición descendente de la cadena.
    Por lo tanto, \(S\) debe tener un elemento minimal.
    Concluímos que \(A\) satisface la condición minimal sobre submódulos.
  \end{proof}
  La prueba para las condiciones de la cadena ascendente y maximal es análoga.
  \begin{proof}
    Supongamos que el módulo \(A\) satisface la condición maximal sobre submódulos y que \(A_1 \subseteq A_2 \subseteq \cdots\) es una cadena de submódulos.
    Entonces el conjunto \(\{A_i : i \geq 1\}\) tiene un elemento maximal, digamos \(A_n\).
    Consecuentemente, para \(i \geq n\) tenemos \(A_n \subseteq A_i\) por hipótesis y \(A_n \supseteq A_i\) por maximalidad.
    Luego \(A_i = A_n\) para todo \(i \geq n\).
    Por lo tanto, \(A\) satisface la condición ascendente de la cadena.

    Recíprocamente, supongamos que \(A\) satisface la condición de la cadena ascendente, y \(S\) es un conjunto novacío de submódulos de \(A\).
    Entonces existe \(B_0 \in S\).
    Si \(S\) no tiene elemento maximal, entonces para todo submódulo \(B \in S\) existe al menos un submódulo \(B' \in S\) tal que \(B \subset B'\).
    Para cada \(B \in S\), elegimos uno de estos \(B'\) (Axioma de Elección).
    Esta elección define una función \(f : S \rightarrow S\) mediante \(B \mapsto B'\).
    Por el Teorema de la Recursión, existe una función \(\phi : \naturalNumbers \rightarrow S\) tal que \(\phi(0) = B_0\) y \(\phi(n + 1) = f(\phi(n))\) \((\forall n \in \naturalNumbers)\).
    Por tanto si \(B_n = \phi(n)\) \((\forall n \in \naturalNumbers)\) entonces \(B_0 \supset B_1 \supset \cdots\) es una cadena ascendente que viola la condición ascendente de la cadena.
    Por lo tanto, \(S\) debe tener un elemento maximal.
    Concluimos que \(A\) satisface la condición maximal de la cadena.
  \end{proof}

  \section{series subnormales}

  Una \emph{serie normal} para un módulo \(A\) es una cadena de submódulos:
  \(A = A_0 \supseteq A_1 \supseteq A_2 \supseteq \cdots \supseteq A_n\).
  Los \emph{factores} de la serie son los módulos cociente \(A_i / A_{i + 1}\) \((0 \leq i < n)\).
  La \emph{longitud} de la serie es el número de inclusiones propias (igual al número de factores notriviales).
  Un \emph{refinamiento propio} es un refinamiento con longitud mayor a la serie original.
  Dos series normales son \emph{equivalentes} si existe una correspondencia uno-a-uno entre los factores notriviales tal que factores correspondientes sean isomorfos.
  De tal modo, series equivalentes tienen igual longitud.
  Una \emph{serie de composición} para \(A\) es una serie normal
  \(A = A_0 \supseteq A_1 \supseteq A_2 \supseteq \cdots \supseteq A_n = 0\)
  tal que cada factor \(A_k / A_{k + 1}\) \((0 \leq k < n)\) es un módulo nonulo sin submódulos propios.
  Si \(R\) es unitario, decimos que un módulo unitario sin submódulos propios es \emph{simple}.

  La Teoría de Series Normales y Subnormales para grupos puede trasladarse al caso de los módulos.
  Esta teoría incluye análogos al lema de Zassenhaus, y a los teoremas de Schreier y Jordan-Hölder.
  Como consecuencia tenemos el siguiente teorema.
  \begin{theorem}
    \label{theorem:refinementAndEquivalenceOfNormalSeries}
    Cualesquiera dos series normales de un módulo \(A\) tienen refinamientos que son equivalentes.
    Caulesquiera dos series de composición de \(A\) son equivalentes.
  \end{theorem}


  \begin{theorem}
    \label{theorem:compositionSeriesAndChainConditions}
    Un módulo nonulo \(A\) tiene una serie de composición si y solo si \(A\) satisface tanto la condición de la cadena descendente como la ascendente.
  \end{theorem}
  \begin{proof}
    Supongamos que \(A\) tiene una serie de composición \(S\) de longitud \(n\).
    Si alguna de las condiciones de la cadena falla, podemos encontrar submódulos \(A = A_0 \supset A_1 \supset A_2 \supset \cdots \supset A_n \supset A_{n + 1}\)
    que forman una serie normal \(T\) de longitud \(n + 1\).
    Por el teorema \ref{theorem:refinementAndEquivalenceOfNormalSeries}, \(S\) y \(T\) tienen refinamientos equivalentes.
    Esto es una contradicción porque series equivalentes tienen igual longitud.
    Sin embargo todo refinamiento de la serie de composición \(S\) tiene longitud \(n\) al igual que \(S\), pero todo refinamiento de \(T\) tiene longitud al menos \(n + 1\).
    Por lo tanto \(A\) satisface ambas condiciones de la cadena.

    Recíprocamente, suponemos que \(A\) satisface ambas condiciones de la cadena.
    Dado \(B\), un submódulo nonulo de \(A\), definimos \(S(B)\) como el conjunto formado por todos los submódulos \(C\) de \(B\) con \(C \neq B\).
    De tal modo que si \(B\) no tiene submódulos propios, entonces \(S(B) = \{0\}\).
    También definimos \(S(0) = \{0\}\).
    Para cada \(B\), la condición de la cadena ascendente nos asegura que el conjunto \(S(B)\) tiene un elemento maximal \(B'\) (por el Teorema \ref{theorem:equivalenceOfChainConditions}
    ).
    Sea \(S\) el conjunto de todos los submódulos de \(A\).
    Definimos una aplicación \(f : S \rightarrow S\) mediante \(f(B) = B'\) (usando el Axioma de Elección).
    Por el Teorema de la Recursión, existe una función \(\phi : \naturalNumbers \rightarrow S\) tal que \(\phi(0) = A\) y \(\phi(n + 1) = f(\phi(n))\).
    Si \(A_i = \phi(i)\), entonces \(A \supseteq A_1 \supseteq A_2 \supseteq \cdots\) es una cadena descendente por construcción.
    Luego, por la condición de la cadena descendente, para un \(n\), \(A_i = A_n\) \((\forall i \geq n)\).
    Dado que \(A_{n + 1} = f(A_n)\), la definición de \(f\) muestra que \(A_{n + 1} = A_n\) solo si \(A_n = 0 = A_{n + 1}\).
    Sea \(m\) el menor entero tal que \(A_m = 0\).
    Entonces \(m \leq n\) y \(A_k \neq 0\) \((\forall k < m)\).
    Más aún, para cada \(k < m\), \(A_{k + 1}\) es por construcción un submódulo maximal de \(A_k\).
    Consecuentemente, cada factor \(A_k / A_{k + 1}\) es nonulo y no tiene submódulos propios por el teorema \ref{theorem:submodulesOfQuotientModule}.
    Por lo tanto \(A \supseteq A_1 \supseteq \cdots \supseteq A_m = 0\) es una serie de composición para \(A\).
  \end{proof}

  \section{Conclusión}

  \begin{corollary}
    \label{corollary:matrixRingIsArtinianAndNoetherian}
    Si \(D\) es un anillo de división, entonces el anillo \(\Mat_n D\) formado por todas las matrices \(n \times n\) sobre \(D\) es a la vez artiniano y noetheriano.
  \end{corollary}
  \begin{proof}
    Es una consecuencia del teorema \ref{theorem:compositionSeriesAndChainConditions}.
    En efecto, usando tal resultado, el teorema \ref{theorem:compositionSeriesOfLeftModulesForMatrixRing} implica que \(\Mat_n D\) es artiniano y noetheriano izquierdo; y el teorema \ref{theorem:compositionSeriesOfRightModulesForMatrixRing} implica que \(\Mat_n D\) es artiniano y noetheriano derecho.
  \end{proof}

  \chapter{Teorema de Artin-Wedderburn}

  \section{Ideales del anillo de matrices}

  El Teorema de Artin-Wedderburn nos dice que el anillo de matrices cuadradas es especial.
  Por eso nos introducimos en este capítulo con el estudio de ideales formados por matrices.
  El primer teorema busca una forma que tiene dicho anillo de ser especial, y nos da la forma precisa que toman sus ideales.


  \begin{theorem}
    \label{theorem:matrixRingIdeals}
    Sea \(R\) un anillo con identidad.
    Entonces \(J\) es un ideal de \(\Mat_n R\) si y solo si \(J = \Mat_n I\) para algún ideal \(I\) en \(R\).
  \end{theorem}
  \begin{proof}
    Escribimos \(E = E_{1, 1}\), \(S = \Mat_n R\).

    Sea \(J\) un ideal de \(S\).
    Sea \(I\) el conjunto formado por todos los elementos de \(R\) que aparecen como entrada \((1, 1)\) de alguna matriz en \(J\).
    Afirmamos que, para todo \(a \in R\), \(a \in I\) si y solo si \(a E \in J\).
    En efecto, si \(a E \in J\) donde \(a \in R\), entonces \(a = (a E)_{1, 1} \in I\).
    Recíprocamente, si \(a \in I\), entonces existe \(A = (a_{i j})\) en \(J\) con \(a_{1 1} = a\).
    Al ser \(J\) un ideal (bilátero), tenemos \(E A E \in J\).
    Pero \(E A E = a E\).
    Entonces \(a E \in J\).

    Afirmamos que \(I\) es un ideal.
    En efecto, \(0_S \in J\) porque \(J\) es un ideal.
    Luego \(0_R = (0_S)_{1, 1} \in I\).
    Por otra parte, si \(a, b \in I\), entonces \(a E, b E \in J\).
    Pero \(J\) es un ideal.
    Entonces \((a + b) E = a E + b E \in J\).
    Luego \(a + b \in I\).
    Para finalizar consideramos \(r \in R\) y \(a \in I\).
    Entonces \(r E \in S\) y \(a E \in J\).
    Pero \(J\) es un ideal.
    Entonces
    \begin{align}
      (r a) E
      &=
      (r a) E^2
      =
      (r E) (a E)
      \in J
      \\
      (a r) E
      &=
      (a r) E^2
      =
      (a E) (r E)
      \in J
    \end{align}
    Luego \(ra, ar \in I\).

    Afirmamos que \(M_n I = J\).
    Sea \(A = (a_{i j})_{i j}\) una matriz en \(S\).
    Comenzamos suponiendo \(A \in J\).
    Consideremos \(i, j \in \{1, \dots, n\}\).
    Porque \(J\) es un ideal (bilátero), \(a_{i j} E = E_{1 i} A E_{j 1} \in J\).
    Luego \(a_{i j} \in I\).
    Porque \(i, j\) eran arbitrarios, se deduce \(A \in M_n I\).
    Recíprocamente, suponemos que \(A \in M_n I\).
    Consideramos \(i, j \in \{1, \dots, n\}\).
    Por hipótesis \(a_{i j} \in I\).
    Luego \(a_{i j} E \in J\).
    Porque \(J\) es un ideal, se deduce \(E_{i 1} (a_{i j} E) E_{1 j} \in J\) mientras \(E_{i 1} (a_{i j} E) E_{1 j} = a_{i j} E_{i j}\).
    Porque \(i, j\) eran arbitrarios, usando que \(J\) está cerrado bajo suma, se deduce \(A = \sum_{i j} a_{i j} E_{i j} \in J\).
  \end{proof}

  Especializando el teorema \ref{theorem:matrixRingIdeals} al caso que es relevante para nosotros, el de los anillos de división, llegamos a uno de los cuatro pilares del Teorema de Artin-Wedderburn.
  Un teorema que conecta un elemento concreto, las matrices, con una cierta organización abstracta de su familia de ideales.

  \begin{theorem}
    \label{theorem:idealsInTheRingOfMatricesOverADivisionRing}
    El anillo \(\Mat_n D\) formado por todas las matrices \(n \times n\) con entradas en un anillo de división \(D\) no tiene ideales propios.
  \end{theorem}
  \begin{proof}
    Si \(J\) es un ideal de \(S\), entonces, por el teorema \ref{theorem:matrixRingIdeals}, \(J = \Mat_n I\) para algún ideal \(I\) en \(D\).
    Pero \(D\) es un anillo de división, no tiene ideales propios.
    Luego \(I = 0\) o \(I = D\), concluyendo que \(J = 0\) o \(J = S\).
  \end{proof}

  \section{Anillos simples y primitivos}

  \begin{definition}
    Un módulo (izquierdo) \(A\) sobre un anillo \(R\) es \emph{simple} (o \emph{irreducible}) si \(R A \neq 0\) y \(A\) no tiene submódulos propios.
    Un anillo \(R\) es \emph{simple} si \(R^2 \neq 0\) y \(R\) no tiene ideales (bilaterales) propios.
  \end{definition}

  \begin{remark}
    El teorema \ref{theorem:idealsInTheRingOfMatricesOverADivisionRing} dice que \(\Mat_n D\) es simple si \(D\) es un anillo de división.
  \end{remark}

  \begin{proposition}
    \label{proposition:simpleModulesAreCiclic}
    Todo \(R\)-módulo simple \(A\) es cíclico; de hecho, \(A = R a\) para todo \(a \in A\) nonulo.
  \end{proposition}
  \begin{proof}
    Sea \(a \in A \setminus 0\).
    Queremos probar \(R a = A\).
    En primer lugar, \(R a\) es un submódulo de \(A\).
    Por simplicidad de \(A\), se deduce que \(R a = A\) o \(R a = 0\).
    Para descartar esta segunda posibilidad, consideramos el conjunto \(B = \{c \in A : R c = 0\}\).
    Resulta que \(B\) también se un submódulo de \(A\).
    Por simplicidad de \(A\), se deduce que \(B = A\) o \(B = 0\).
    Pero la simplicidad de \(A\) también implica \(R A \neq 0\), mientras que \(R B = 0\).
    Luego \(B \neq A\) y \(B = 0\).
    Por lo tanto \(a \notin B\) y \(R a \neq 0\).
    En conclusión \(R a = A\).
  \end{proof}

  \begin{definition}
    Sea \(B\) un subconjunto de un mádulo izquierdo \(A\) sobre un anillo \(R\).
    Definimos el \emph{aniquilador (izquierdo)} de \(B\) como \(\mathcal{A}(B) = \{r \in R : r B = 0\}\).
  \end{definition}

  El aniquilador derecho de un módulo derecho se define análogamente.

  \begin{theorem}
    \label{theorem:anihilatorTheorem}
    Sea \(B\) un subconjunto de un módulo izquierdo \(A\) sobre un anillo \(R\).
    Entonces \(\mathcal{A}(B)\) es un ideal izquierdo de \(R\).
    Si \(B\) es un submódulo de \(A\), entonces \(\mathcal{A}(B)\) es un ideal bilátero.
  \end{theorem}

  \begin{definition}
    Un módulo (izquierdo) \(A\) es \emph{fiel} si \(\mathcal{A}(A) = 0\).
    Un anillo \(R\) es \emph{primitivo} (\emph{izquierdo}) si existe un \(R\)-módulo simple y fiel.
  \end{definition}

  \begin{theorem}
    \label{theorem:simpleLeftArtinianRingsArePrimitive}
    Un anillo artiniano izquierdo es primitivo si es simple.
  \end{theorem}
  \begin{proof}
    Sea \(R\) un anillo artiniano izquierdo simple.

    Dado que \(R\) es artiniano izquierdo, el conjunto formado por todos sus ideales izquierdos nonulos contiene un ideal izquierdo minimal \(J\) (una consecuencia del teorema \ref{theorem:equivalenceOfChainConditions}).
    Esta minimalidad hace que \(J\), como \(R\)-módulo, no tenga \(R\)--submódulos propios (un \(R\)--submódulo de \(J\) es un ideal izquierdo de \(R\) contenido en \(J\)).

    Afirmamos que \(R J \neq 0\).
    Para probarlo consideramos \(I = \{r \in R :  R r = 0\}\), notando que \(R J = 0\) si y solo si \(J \subseteq I\).
    Vemos que \(I\) es un ideal de \(R\).
    Por simplicidad de \(R\), se deduce \(I = R\) o \(I = 0\).
    Pero \(R I = 0\), y la simplicidad de \(R\) también implica que \(R R \neq 0\).
    Luego \(I \neq R\) e \(I = 0\).
    Es decir que \(R J = 0\) solo si \(J = 0\).
    Por construcción \(J \neq 0\).
    En conclusión \(R J \neq 0\).

    Afirmamos que el aniquilador izquierdo \(\mathcal{A}(J)\) de \(J\) en \(R\) es cero.
    Primero que nada, \(\mathcal{A}(J)\) es un ideal bilátero de \(R\) porque \(J\) es un ideal izquierdo.
    Luego, la simplicidad de \(R\) implica que \(\mathcal{A}(J) = R\) o \(\mathcal{A}(J) = 0\).
    Si fuera \(\mathcal{A}(J) = R\), tendríamos \(R J = 0\).
    Pero probamos \(R J \neq 0\).
    Por lo tanto \(\mathcal{A}(J) \neq R\) y \(\mathcal{A}(J) = 0\).

    Juntos, los dos primeros resultados nos dicen que \(J\) es un \(R\)-módulo simple.
    El tercer resultado nos dice que \(J\) es fiel.
    Luego, la existencia de \(J\) garantiza que \(R\) es primitivo.
  \end{proof}

  \section{El Teorema de Densidad de Jacobson}

  Sea \(R\) un anillo y \(A\) un \(R\)-módulo 
  ¿Qué hace un elemento \(r \in R\) en esta relación?
  Multiplica elementos \(a \in A\), resultando \(r a \in A\).
  Es decir que cada \(r \in R\) define una función \(m_r : A \rightarrow A\) dada por \(a \mapsto r a\).
  Esta función puede no ser un \(R\)-endomorfismo de \(A\), por ejemplo si \(R\) no es conmutativo.
  A pesar de ello, si pensamos \(A\) como un módulo sobre el anillo \(E = \End_R A\), entonces \(m_r\) es un \(E\)-endomorfismo de \(A\): esto es, \(m_r \in \End_E A\).
  Aquí no se termina la tela para cortar.
  Tenemos \(m_{r + s} = m_r + m_s\) y \(m_{r s} = m_r \circ m_s\) para todo par \(r, s \in R\).
  Consecuentemente, la aplicación \(m : R \rightarrow \End_E A\) definida por \(r \mapsto m_r\) es un homomorfismo de anillos.
  Llegamos a un nuevo anillo \(\mathcal{M} = \image(m)\), al que llamamos \emph{anillo de multiplicadores}.

  Dados los anillos \(R\) y \(\mathcal{M}\), junto al homomorfismo \(m : R \rightarrow \mathcal{M}\) que los une, surge la pregunta
  ¿cuándo es \(m\) un isomorfismo?
  La respuesta es que si el \(R\)-módulo \(A\) es fiel, entonces \(m\) es un isomorfismo entre los anillos \(R\) y \(\mathcal{M}\).

  \begin{theorem}
    \label{theorem:multiplierIsomorphism}
    Sea \(R\) un anillo, \(A\) un \(R\)-módulo y \(\mathcal{M}\) el anillo de multiplicadores.
    Si \(A\) es fiel, entonces \(m : R \rightarrow \mathcal{M}\) es un isomorfismo.
  \end{theorem}
  \begin{proof}
    Sea \(r \in R\).
    Dado que \(A\) es un \(R\)-módulo fiel, \(m_r = 0\) si y solo si \(r \in \mathcal{A}(A) = 0\).
    De aquí que \(m\) es un monomorfismo, y \(R\) es isomorfo al anillo \(\mathcal{M} = \image(m)\).
  \end{proof}

  Vemos que el anillo \(\mathcal{M}\) se ve afectado por la hipótesis de que \(A\) es fiel
  ¿Qué sucedería si \(A\) fuera simple?
  Una primera aproximación nos muestra que este es un caso especial.
  Si \(A\) es simple, entonces \(D = \End_R A\) es un anillo de división y \(A\) es un espacio vectorial sobre \(D\).
  La respuesta completa es el resultado más denso de la sección, y resulta ser un lema fundamental para llegar al Teorema de Densidad de Jacobson.
  Afirma la existencia de una biyección entre los subespacios finito-dimensionales de \(A\) y sus aniquiladores.

  \begin{lemma}
    \label{lemma:anihilatorLemma}
    Sea \(R\) un anillo y \(A\) un \(R\)-módulo simple.
    Si \(V\) es un \(D\)-subespacio finito-dimensional de \(A\) entonces \(V = \{a \in A : \mathcal{A}(V) a = 0\}\).
  \end{lemma}
  \begin{proof}
    Por inducción sobre \(n = \dim_D V\). 
    Comenzamos por el caso base.
    Si \(n = 0\), entonces \(V = 0\) y \(\mathcal{A}(V) = R\).
    Por simplicidad de \(A\), para todo \(a \in A\) la ecuación \(R a = 0\) implica \(a = 0\).
    Luego \(0 = \{a \in A : R a = 0\}\), como queríamos.

    Supongamos \(n > 0\), y que el teorema es verdadero para dimensiones menores a \(n\).
    Sea \(\{w_1, \dots, w_{n - 1}, u\}\) una \(D\)-base de \(V\) y sea \(W\) el subespacio \((n - 1)\)-dimensional de \(V\) generado por \(\{w_1, \dots, w_{n - 1}\}\) (siendo \(W = 0\) cuando \(n = 1\)).
    Entonces \(V = W \oplus D u\) (suma directa) y (en consecuencia) \(\mathcal{A}(V) = \mathcal{A}(W) \cap \mathcal{A}(D u)\).
    Sea \(a \in A\) tal que \(\mathcal{A}(V) a = 0\).
    Definimos una relación \(\theta : A \rightarrow A\) como \(r u \mapsto r a\) para todo \(r \in \mathcal{A}(W)\).
    Su dominio es el submódulo de \(A\) definido como \(\mathcal{A}(W) u\).
    Este submódulo es nonulo porque \(u \notin W\) y nuestra hipótesis inductiva garantiza la existencia de un \(r \in \mathcal{A}(W)\) con \(r u \neq 0\).
    Pero \(A\) es simple, y por lo tanto \(\mathcal{A}(W) u = A\).
    Además de tener un dominio adecuado, es una función bien definida porque si \(r u = s u\) para \(r, s \in \mathcal{A}(W)\), entonces \(r - s \in \mathcal{A}(W)\) y \((r - s) u = 0\), lo que implica \(r - s \in \mathcal{A}(V)\).
    Pero \(\mathcal{A}(V) a = 0\), y por lo tanto \(r a = s a\).
    Podemos probar que \(\theta\) es un homomorfismo de \(R\)-módulos, un elemento de \(D\).
    Por eso, para todo \(r \in \mathcal{A}(W)\) podemos calcular \(r (\theta(u) - a) = 0\).
    Luego \( \mathcal{A}(W) (\theta u - a) = 0\).
    Por hipótesis inductiva \(\theta u - a = - w \in W\).
    Luego \(a = w + \theta u\) es un elemento de \(V\), como queríamos.
  \end{proof}

  \begin{definition}
    Sea \(V\) un espacio vectorial izquierdo sobre un anillo de división \(D\).
    Un subanillo \(R\) del anillo de endomorfismos \(\Hom_D(V, V)\) es un \emph{anillo denso de endomorfismos} de \(V\) (o un \emph{subanillo denso} de \(\Hom_D(V, V)\)) si para todo entero positivo \(n\), cada subconjunto linealmente independiente \(\{u_1, \dots, u_n\}\) de \(V\) y cada subconjunto arbitrario \(\{v_1, \dots, v_n\}\) de \(V\), existe \(\theta \in R\) tal que \(\theta(u_i) = v_i\) \((\forall i \in \{1, \dots, n\})\).
  \end{definition}

  \begin{theorem}
    \label{theorem:denseMultiplierRing}
    Sea \(R\) un anillo, y \(A\) un \(R\)-módulo simple considerado a la vez como espacio vectorial sobre \(D = \End_R A\).
    El anillo de multiplicadores \(\mathcal{M}\) es un subanillo denso de endomorfismos de \(A\).
  \end{theorem}
  \begin{proof}
    Sea \(U = \{u_1, \dots, u_n\}\) un subconjunto \(D\)-linealmente independiente de \(A\);
    y sea \(\{v_1, \dots, v_n\}\) un subconjunto arbitrario de \(A\).
    Debemos encontrar \(m_r \in \mathcal{M}\) tal que \(m_r(u_i) = v_i\) \((\forall i \in \{1, \dots, n\})\).
    Para cada \(i\) sea \(V_i\) el \(D\)-subespacio de \(A\) generado por \(\{u_j : j \neq i\}\).
    Dado que \(U\) es \(D\)-linealmente independiente, \(u_i \notin V_i\).
    Por el lema \ref{lemma:anihilatorLemma} existen \(r_i \in \mathcal{A}(V_i)\) tales que \(r_i u_i \neq 0\).
    La simplicidad de \(A\), por la proposición \ref{proposition:simpleModulesAreCiclic}, implica \(R (r_i u_i) = A\).
    Por esto existen \(t_i \in R\) tales que \(t_i r_i u_i = v_i\).
    Sea \(r = t_1 r_1 + \cdots + t_n r_n \in R\).
    Recordar que \(u_i \in V_j\) para \(i \neq j\), luego \(t_j r_j u_i \in t_j (r_j V_j) = t_j 0 = 0\).
    Consecuentemente \(m_r(u_i) = (t_1 r_1 + \cdots + t_n r_n) u_i = t_i r_i u_i = v_i\), tal como queríamos.
  \end{proof}

  \begin{theorem}[de Densidad de Jacobson]
    \label{theorem:jacobsonDensityTheorem}
    Sea \(R\) un anillo y \(A\) un \(R\)-módulo simple y fiel.
    Considerar \(A\) como espacio vectorial sobre el anillo de división \(D = \End_R A\).
    Entonces \(R\) es isomorfo al anillo de multiplicadores \(\mathcal{M}\), y este es un anillo denso de endomorfismos de \(A\).
  \end{theorem}
  \begin{proof}
    Dado que \(A\) es un \(R\)-módulo fiel, el teorema \ref{theorem:multiplierIsomorphism} implica que \(R \simeq \mathcal{M}\).
    Además, la simplicidad de \(A\), mediante el teorema \ref{theorem:denseMultiplierRing}, implica que \(\mathcal{M}\) es un subanillo denso de \(\End_D A\).
  \end{proof}

  \begin{theorem}
    \label{theorem:leftArtinianDenseEndomorphismRingsOfVectorSpaces}
    Sea \(R\) un anillo denso de endomorfismos de un espacio vectorial \(V\) sobre un anillo de división \(D\).
    Entonces \(R\) es artiniano izquierdo [resp. derecho] si y solo si \(\dim_D V\) es finita, en cuyo caso \(R = \Hom_D(V, V)\).
  \end{theorem}
  \begin{proof}
    Supongamos \(\dim_D V\) es infinita.
    Entonces existe un subconjunto de \(V\) linealmente independiente e infinito (numerable) \(\{u_1, u_2, \dots\}\).
    Como \(V\) es un \(\Hom_D(V, V)\)-módulo izquierdo, también es un \(R\)-módulo izquierdo (recordar que \(R \subseteq \Hom_D(V, V)\)).
    Para cada \(n\) sea \(I_n = \mathcal{A}(\{u_1, \dots, u_n\})\) el aniquilador izquierdo en \(R\) del conjunto \(\{u_1, \dots, u_n\}\).
    Por el Teorema \ref{theorem:anihilatorTheorem}, la cadena descencendente \(I_1 \supseteq I_2 \supseteq \cdots\) está compuesta de ideales izquierdos de \(R\).
    Sea \(w\) un elemento nonulo de \(V\), no importa cual de ellos sea (podría ser \(u_1\), por ejemplo).
    Dado que \(\{u_1, \dots, u_{n + 1}\}\) es linealmente independiente (para cada \(n\)) y \(R\) es denso, existe \(\theta \in R\) tal que \(\theta u_i = 0\) \((\forall i \in \{1, \dots, n\})\) y \(\theta u_{n + 1} = w \neq 0\).
    Consecuentemente \(\theta \in I_n\) pero \(\theta \notin I_{n + 1}\).
    Por lo tanto \(I_1 \supset I_2 \supset \cdots\) es una cadena estrictamente descendente de \(R\)-módulos izquierdos.
    La existencia de esta cadena dice que \(R\) no es artiniano.

    Supongamos que \(\dim_D V\) es finita.
    En este caso \(V\) tiene una base finita \(\{v_1, \dots, v_m\}\).
    Si \(f\) es un elemento de \(\Hom_D(V, V)\), entonces \(f\) está completamente determinado por su acción sobre \(v_1, \dots, v_m\) por los teoremas \ref{theorem:freeUnitalModulesOverARingWithIdentity} y \ref{theorem:vectorSpaceBasis}.
    Dado que \(R\) es denso, existe \(\theta \in R\) tal que \(\theta v_i = f v_i\) \(\forall i \in \{1, \dots, m\}\).
    Luego \(f = \theta \in R\).
    Por lo tanto \(\Hom_D(V, V) = R\).
    Pero \(\Hom_D(V, V)\) es artiniano por el Teorema \ref{theorem:isomorphismBetweenMatrixAndHomomorphismRings} y el corolario \ref{corollary:matrixRingIsArtinianAndNoetherian}.
  \end{proof}

  \begin{corollary}
    \label{corollary:leftArtinianPrimitiveRingsAreRingsOfEndomorphisms}
    Un anillo \(R\) artiniano izquierdo y primitivo es isomorfo al anillo de endomorfismos de un espacio vectorial nonulo sobre un anillo de división \(D\).
  \end{corollary}
  \begin{proof}
    Por el Teorema de Densidad de Jacobson \ref{theorem:jacobsonDensityTheorem}, \(R\) es isomorfo a un anillo denso \(T\) compuesto por endomorfismos de un espacio vectorial \(V\) sobre un anillo de división \(D\).
    Porque \(R\) es artiniano izquierdo, el Teorema \ref{theorem:leftArtinianDenseEndomorphismRingsOfVectorSpaces} implica \(R \simeq T = \Hom _D(V, V)\).
  \end{proof}

  \begin{theorem}[de Artin--Wedderburn]
    Las siguientes condiciones sobre un anillo artiniano izquierdo \(R\) son equivalentes.
    \begin{enumerate}
      \item
        \label{property:artin--wedderburn_simple-ring}
        \(R\) es simple;
      \item
        \label{property:artin--wedderburn_primitive-ring}
        \(R\) es primitivo;
      \item
        \label{property:artin--wedderburn_endomorphism-ring}
        \(R\) es isomorfo al anillo de endomorfismos de un espacio vectorial nonulo sobre un anillo de división \(D\);
      \item
        \label{property:artin--wedderburn_matrix-ring}
        para algún entero positivo \(n\), \(R\) es isomorfo al anillo formado por las matrices \(n \times n\) sobre un anillo de división.
    \end{enumerate}
  \end{theorem}
  \begin{proof}
    \(\ref{property:artin--wedderburn_simple-ring} \Rightarrow \ref{property:artin--wedderburn_primitive-ring}\).
    Teorema \ref{theorem:simpleLeftArtinianRingsArePrimitive}

    \(\ref{property:artin--wedderburn_primitive-ring} \Rightarrow \ref{property:artin--wedderburn_endomorphism-ring}\)
    Corolario \ref{corollary:leftArtinianPrimitiveRingsAreRingsOfEndomorphisms}.

    \(\ref{property:artin--wedderburn_endomorphism-ring} \Leftrightarrow \ref{property:artin--wedderburn_matrix-ring}\)
    Teorema \ref{theorem:isomorphismBetweenMatrixAndHomomorphismRings}

    \(\ref{property:artin--wedderburn_matrix-ring} \Rightarrow \ref{property:artin--wedderburn_simple-ring}\)
    Teorema \ref{theorem:idealsInTheRingOfMatricesOverADivisionRing}
  \end{proof}

  \newpage

  \begin{theorem}
    \label{theorem:isomorphismBetweenMatrixAndHomomorphismRings}
    Sea \(R\) un anillo con identidad y \(E\) un \(R\)-módulo izquierdo libre con una base finita de \(n\) elementos.
    Entonces existe un isomorfismo de anillos
    \begin{align}
      \Hom_R(E, E)
      \simeq
      \Mat_n(R^{\textnormal{op}})
    \end{align}
    En particular, este isomorfismo existe para todo espacio vectorial \(E\) sobre un anillo de división \(R\) con dimensión \(n\), en cuyo caso \(R^{\textnormal{op}}\) también es un anillo de división.
  \end{theorem}
  \begin{remark}
    Cuando \(R\) es conmutativo \(R = R^{\textnormal{op}}\).
    La fórmula del teorema resulta \(\Hom_R(E, E) \simeq \Mat_n R\).
  \end{remark}
\end{document}