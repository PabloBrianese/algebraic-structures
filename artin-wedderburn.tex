\documentclass{report}
\usepackage[spanish]{babel}

\usepackage{mathtools}
\usepackage{amsthm, amsfonts}

\newcommand{\naturalNumbers}{\mathbb{N}}
\newcommand{\Hom}{\textnormal{Hom}}
\newcommand{\Mat}{\textnormal{Mat}}
\newcommand{\Col}{\textnormal{Col}}
\newcommand{\Fila}{\textnormal{Fila}}

\DeclareMathOperator{\image}{\text{Im}}

\newtheorem{theorem}{Teorema}
\newtheorem{lemma}{Lema}
\newtheorem{proposition}{Proposición}
\newtheorem{definition}{Definición}
\newtheorem{remark}{Observación}

\title{Teorema de Artin--Wedderburn}
\author{Pablo Brianese}
\begin{document}
  \maketitle

  \begin{theorem}[Lema de Zorn]
    Si \(A\) es un conjunto parcialmente ordenado novacío tal que toda cadena en \(A\) tiene una cota superior en \(A\), entonces \(A\) contiene un elemento maximal.
  \end{theorem}

  \begin{theorem}
    \label{theorem:submodulesOfQuotientModule}
    Si \(R\) es un anillo y \(B\) es un submódulo de un \(R\)-módulo \(A\), entonces existe una correspondencia uno-a-uno entre el conjunto de los submódulos de \(A\) que contienen a \(B\) y el conjunto de todos los submódulos de \(A / B\), dada por \(C \mapsto C / B\).
    Por tanto todo submódulo de \(A / B\) es de la forma \(C / B\), donde \(C\) es un submódulo de \(A\) que contiene a \(B\).
  \end{theorem}

  \begin{theorem}
    \label{theorem:freeUnitalModulesOverARingWithIdentity}
    Sea \(R\) un anillo con identidad.
    Las siguientes condiciones sobre un \(R\)-módulo unitario \(F\) son equivalentes:
    \begin{enumerate}
      \item \(F\) tiene una base novacía;
      \item \(F\) es la suma directa (interna) de una famile de \(R\)-módulos cíclicos, cada uno de los cuales es isomorfo (como \(R\)-módulo izquierdo) a \(R\);
      \item \(F\) es isomorfo (como \(R\)-módulo) a una suma directa de copias del \(R\)-módulo izquierdo \(R\);
      \item existe un conjunto novacío \(X\) y una función \(\imath : X \rightarrow F\) con la siguiente propiedad:
      dado un \(R\)-módulo unitario \(A\) y una función \(f : X \rightarrow A\), existe un único homomorfismo de \(R\)-módulos \(\bar{f} : F \rightarrow A\) tal que \(\bar{f} \imath = f\).
      En otras palabras, \(F\) es un objeto libre en la categoría de \(R\)-módulos unitarios.
    \end{enumerate}
  \end{theorem}

  Un módulo unitario \(F\) sobre un anillo \(R\) con identidad, que satisface las condiciones del teorema, recibe el nombre de \emph{\(R\)-módulo libre} sobre el conjunto \(X\).
  La cuarta propiedad hace de \(F\) un objeto libre en la categoría formada por los 

  \begin{theorem}
    \label{theorem:vectorSpaceBasis}
    Todo espacio vectorial \(V\) sobre un anillo de división \(D\) tiene una base y es por tanto un \(D\)-módulo libre.
    Con mayor generalidad, cada subconjunto linealmente independiente de \(V\) está contenido en una base de \(V\).
  \end{theorem}

  \begin{theorem}
    Sean \(A\) y \(B\) ambos \(R\)-módulos.
    \begin{enumerate}
      \item el conjunto \(\Hom_R(A, B)\) formado por los homomorfismos de \(R\)-módulos \(A \rightarrow B\) es un grupo abeliano con \(f + g : A \rightarrow B\) dada por \(a \mapsto f(a) + g(a)\).
      El elemento identidad es la aplicación nula.
      \item \(\Hom_R(A, A)\) es un anillo con identidad, donde la multiplicación es la composición de funciones.
      \(\Hom_R(A, A)\) es el \emph{anillo de endomorfismos} de \(A\).
      \item \(A\) es un \(\Hom_R(A, A)\)-módulo izquierdo con \(f a = f(a)\) \((\forall a \in A)\) \((\forall f \in \Hom_R(A, A))\).
    \end{enumerate}
  \end{theorem}

  \begin{theorem}
    \label{theorem:isomorphismBetweenMatrixAndHomomorphismRings}
    Sea \(R\) un anillo con identidad y \(E\) un \(R\)-módulo izquierdo libre con una base finita de \(n\) elementos.
    Entonces existe un isomorfismo de anillos
    \begin{align}
      \Hom_R(E, E)
      \simeq
      \Mat_n(R^{\textnormal{op}})
    \end{align}
    En particular, este isomorfismo existe para todo espacio vectorial \(E\) sobre un anillo de división \(R\) con dimensión \(n\), en cuyo caso \(R^{\textnormal{op}}\) también es un anillo de división.
  \end{theorem}
  \begin{remark}
    Cuando \(R\) es conmutativo \(R = R^{\textnormal{op}}\).
    La fórmula del teorema resulta \(\Hom_R(E, E) \simeq \Mat_n R\).
  \end{remark}

  \begin{proposition}
    Sea \(R\) un anillo con identidad, y \(S\) el anillo formado por todas las matrices \(n \times n\) sobre \(R\).
    Dentro de \(S\) podemos encontrar las matrices \(E_{r s}\), donde \(r, s \in \{1, \dots, n\}\), y \(E_{r s}\) tiene \(1_R\) como entrada \((r, s)\) y 0 en as demás posiciones.
    Para toda matriz \(A = (a_{i j})\) en \(S\)
    \begin{align}
      E_{p r} A E_{s q}
      =
      a_{r s} E_{p q}
    \end{align}
  \end{proposition}
  \begin{proof}
    Es un cálculo directo.
  \end{proof}

  \begin{proposition}
    Si \(D\) es un anillo de división y \(R = \Mat_n D\).
    Entonces, para toda matriz \(A \in R\), \(R A\) es un ideal izquierdo de \(R\) y \(A R\) es un ideal derecho de \(R\).
  \end{proposition}
  \begin{proof}
    No requiere mucho razonamiento, es un cálculo directo.
  \end{proof}

  \begin{theorem}
    Si \(D\) es un anillo de división y \(R = \Mat_n D\), entonces el ideal \(R E_{j_0 j_0}\) está formado por todas las matrices \(A \in R\) tales que \(\Col_j A = 0\) \((\forall j \neq j_0)\).
  \end{theorem}
  \begin{proof}
    Fijemos \(j_0 \in \{1, \dots, n\}\), y escribamos \(E = E_{j_0 j_0}\), \(I = R E\).

    Afirmamos que \(I' = \{A \in R : \Col_j A = 0 \, (\forall j \neq j_0)\}\) es igual a \(I\).
    Lo demostraremos usando que para toda matriz \(a = (a_{i j})_{i j}\) en \(R\)
    \begin{align}
      a E_{j_0 j_0}
      =
      I_n a E_{j_0 j_0}
      =
      \sum_{i = 1}^n E_{i i} a E_{j_0 j_0}
      =
      \sum_{j = 1}^n a_{i j_0} E_{i j_0}
    \end{align}
    Si \(A \in I\), entonces existe \(a \in R\) con \(A = a E\).
    Luego \(A = \sum_{i = 1}^n a_{i j_0} E_{i j_0}\) pertenece a \(I'\).
    Recíprocamente, si \(A \in I'\), entonces \(A = (A_{i j})_{i j}\) puede escribirse como \(A = \sum_{i = 1}^n \sum_{j = 1}^n A_{i j} E_{i j} = \sum_{i = 1}^n A_{i j_0} E_{i j_0} = A E_{j_0 j_0}\).
  \end{proof}

  \begin{theorem}
    Sean \(D\) un anillo de división y \(R = \Mat_n D\).
    Entonces son simples los \(R\)-submódulos izquierdos de \(R\)
    \begin{align}
      R E_{j j}
      &&(j \in \{1, \dots, n\})
    \end{align}
  \end{theorem}
  \begin{proof}
    Fijemos \(j_0 \in \{1, \dots, n\}\), y escribamos \(E = E_{j_0 j_0}\), \(I = R E\).

    Afirmamos que \(I\) es minimal.
    Supongamos que \(J\) es un submódulo nonulo de \(I\).
    Entonces existe \(a \in J \setminus 0\).
    Porque \(a \in I\) se sigue \(a = \sum_{i = 1}^n a_{i j_0} E_{i j_0}\).
    \(a \neq 0\) implica que \(a_{i_0 j_0} \neq 0\) para un \(i_0 \in \{1, \dots, n\}\).
    Porque \(D\) es un anillo de división, existe una matriz elemental de transformación \(M\), que actúa sobre \(a\) multiplicando (por izquierda) su fila \(i_0\) por el elemento \(a_{i_0 j_0}^{- 1} \in D\).
    Entonces \(M a = 1_D E_{i_0 j_0} + \sum_{i \neq i_0} a_{i j_0} E_{i j_0}\).
    Luego, existen matrices elementales de transformación \(A_i\) \((i \in \{1, \dots, n\} \setminus i_0)\), que actúan sobre \(a\) sumando a la fila \(i\)-ésima el producto (por izquierda) de \(- a_{i j_0}\) con la fila \(i_0\)-ésima.
    Entonces \(A_1 \cdots A_n M a = 1_D E_{i_0 j_0}\) (donde definimos \(A_{i_0} = I_n\) para mejorar la notación).
    Finalmente, para cada \(i \in \{1, \dots, n\}\), existe una matriz elemental de transformación \(P_i\) que actúa sobre \(a\) permutando las filas \(i\) e \(i_0\).
    De ese modo \(P_i A_1 \cdots A_n M a = 1_D E_{i j_0}\).
    Por lo tanto \(1_D E_{i j_0} \in J\) para todo \(i \in \{1, \dots, n\}\).
    Eso implica que \(I = \sum_{i = 1}^n D E_{i j_0} \subseteq J\).
    En conclusión \(J = I\).
  \end{proof}

  Argumentos análogos demuestran que
  \begin{theorem}
    Si \(D\) es un anillo de división y \(R = \Mat_n D\), entonces el ideal \(E_{i_0 i_0} R\) está formado por todas las matrices \(A \in R\) tales que \(\Fila_i A = 0\) \((\forall i \neq i_0)\).
  \end{theorem}
  \begin{proof}
    Fijemos \(i_0 \in \{1, \dots, n\}\), y escribamos \(E = E_{i_0 i_0}\), \(I = E R\).

    Afirmamos que \(I' = \{A \in R : \Fila_i A = 0 \, (\forall i \neq i_0)\}\) es igual a \(I\).
    Lo demostraremos usando que para toda matriz \(a = (a_{i j})_{i j}\) en \(R\)
    \begin{align}
      E_{i_0 i_0} a
      =
      E_{i_0 i_0} a I_n
      =
      \sum_{j = 1}^n E_{i_0 i_0} a E_{j j}
      =
      \sum_{j = 1}^n a_{i_0 j} E_{i_0 j}
    \end{align}
    Si \(A \in I\) entonces existe \(a \in R\) con \(A = E a\).
    Luego \(A = \sum_{j = 1}^n a_{i_0 j} E_{i_0 j}\) pertenece a \(I'\).
    Recíprocamente, si \(A \in I'\) entonces \(A = (A_{i j})_{i j}\) puede escribirse como \(A = \sum_{i = 1}^n \sum_{j = 1}^n A_{i j} E_{i j} = \sum_{j = 1}^n A_{i_0 j} E_{i_0 j} = E_{i_0 i_0} A\).
  \end{proof}

  \begin{theorem}
    Sean \(D\) un anillo de división y \(R = \Mat_n D\).
    Entonces son simples los \(R\)-submódulos derechos de \(R\)
    \begin{align}
      E_{i i} R
      &&(i \in \{1, \dots, n\})
    \end{align}
  \end{theorem}
  \begin{proof}
    Fijemos \(i_0 \in \{1, \dots, n\}\), y escribamos \(E = E_{i_0 i_0}\), \(I = E R\).

    Afirmamos que \(I\) es minimal.
    Supongamos que \(J\) es un submódulo nonulo de \(I\).
    Entonces existe \(a \in J \setminus 0\).
    Porque \(a \in I\) se sigue \(a = \sum_{j = 1}^n a_{i_0 j} E_{i_0 j}\).
    \(a \neq 0\) implica que \(a_{i_0 j_0} \neq 0\) para un \(j_0 \in \{1, \dots, n\}\).
    Porque \(D\) es un anillo de división, existe una matriz elemental de transformación \(M\), que actúa sobre \(a\) multiplicando (por derecha) su columna \(j_0\) por el elemento \(a_{i_0 j_0}^{- 1} \in D\).
    Entonces \(a M = E_{i_0 j_0} 1_D + \sum_{j \neq j_0} a_{i_0 j} E_{i_0 j}\).
    Luego, existen matrices elementales de transformación \(A_j\) \((j \in \{1, \dots, n\} \setminus j_0)\), que actúan sobre \(a\) sumando a la columna \(j\)-ésima el producto (por derecha) de \(- a_{i_0 j}\) con la columna \(j_0\)-ésima.
    Entonces \(a M A_1 \cdots A_n = E_{i_0 j_0} 1_D\) (donde definimos \(A_{j_0} = I_n\) para mejorar la notación).
    Finalmente, para cada \(j \in \{1, \dots, n\}\), existe una matriz elemental de transformación \(P_j\) que actúa sobre \(a\) permutando las columnas \(j\) y \(j_0\).
    De ese modo \(a M A_1 \cdots A_n P_j = E_{i_0 j} 1_D\).
    Por lo tanto \(E_{i_0 j} 1_D \in J\) para todo \(i \in \{1, \dots, n\}\).
    Eso implica que \(I = \sum_{j = 1}^n E_{i_0 j} D \subseteq J\).
    En conclusión \(J = I\).
  \end{proof}

  \begin{theorem}
    Sea \(M_0 = 0\) y para \(i \in \{1, \dots, n\}\) sea \(M_i = R (E_{1 1} + \cdots + E_{i i})\).
    Afirmamos que cada \(M_i\) es un ideal izquierdo de \(R\) y que \(M_i / M_{i - 1} \simeq R E_{i i}\).
    Por eso \(R = M_n \supseteq M_{n - 1} \supseteq \cdots \supseteq M_1 \supseteq M_0 = 0\) es una serie de composicón de \(R\)-módulos izquierdos.
  \end{theorem}
  \begin{proof}
    Notar que \(M_i \subseteq R E_{1 1} + \cdots + R E_{i i}\).
    Luego \(\Col_j A = 0\) \((\forall j \in \{i + 1, \dots, n\})\) para toda \(A \in M_i\).

    Notar que si \(A \in M_i\) con \(A = r (E_{1 1} + \cdots + E_{i i})\) para un \(r \in R\), entonces \(A E_{i i} = r E_{i i}\).
    En efecto
    \begin{align}
      A E_{i i}
      &=
      r (E_{11} + \cdots + E_{i i}) E_{i i}
      \\
      &=
      r (E_{11} E_{i i} + \cdots + E_{i - 1, i - 1} E_{i i} + E_{i i}^2)
      \\
      &=
      r (0 + \cdots + 0 + E_{i i})
      \\
      &=
      r E_{i i}
    \end{align}
    Por este motivo \(A + M_{i - 1} = A E_{i i} + M_{i - 1}\).
    Calculamos
    \begin{align}
      A + M_{i - 1}
      &=
      r (E_{1 1} + \cdots + E_{i i}) + M_{i - 1}
      \\
      &=
      r(E_{1 1} + \cdots + E_{i - 1, i - 1}) + r E_{i i} + M_{i - 1}
      \\
      &=
      r E_{i i} + M_{i - 1}
      \\
      &=
      A E_{i i} + M_{i - 1}
    \end{align}

    Supongamos que \(A + M_{i - 1} = B + M_{i - 1}\).
    Entonces \(A E_{i i} + M_{i - 1} = B E_{i i} + M_{i - 1}\).
    Escribamos \(C = (A - B) E_{i i}\).
    Por un lado \(C \in M_{i - 1}\) y por el otro \(C \in R E_{i i}\).
    El primer dato implica \(\Col_j C = 0\) para \(j \in \{i, \dots, n\}\).
    El segundo dato implica \(\Col_j C = 0\) para \(j \in \{1, \dots, n\} \setminus i\).
    Luego \(C = 0\).
    Es decir \(A E_{i i} = B E_{i i}\).

    Esto nos permite definir una función \(\phi : M_i / M_{i - 1} \rightarrow R E_{i i}\) dada por \(A + M_{i - 1} \mapsto A E_{i i}\).
    Así definida, es un homomorfismo de \(R\)-módulos.

    Es además un monomorfismo.
    Si \(\phi(A + M_{i - 1}) = 0\) entonces \(A E_{i i} = 0\) y \(\Col_i A = 0\).
    Además, \(A \in M_i\) implica \(\Col_j A = 0\) \((\forall j > i)\).
    Luego \(\Col_j A = 0\) \((\forall j \in \{i, \dots, n\})\), y \(A \in M_{i - 1}\).
    Entonces \(A + M_{i - 1} = 0 + M_{i - 1}\).

    También es un epimorfismo.
    Dado \(A E_{i i} \in R E_{i i}\), tenemos \(A E_{i i} \in M_i\).
    Calculamos \(\phi(A E_{i i} + M_{i - 1})= (A E_{i i}) E_{i i} = A E_{i i}^2 = A E_{i i}\).
    Por lo tanto \(A E_{i i} \in \image \phi\).

    Concluímos que \(\phi : M_i / M_{i - 1} \rightarrow R E_{i i}\) es un isomorfismo.
  \end{proof}

  \begin{theorem}
    Sea \(M_0 = 0\) y para \(i \in \{1, \dots, n\}\) sea \(M_i = (E_{1 1} + \cdots + E_{i i}) R\).
    Afirmamos que cada \(M_i\) es un ideal derecho de \(R\) y que \(M_i / M_{i - 1} \simeq E_{i i} R\).
    Por eso \(R = M_n \supseteq M_{n - 1} \supseteq \cdots \supseteq M_1 \supseteq M_0 = 0\) es una serie de composición de \(R\)-módulos derechos.
  \end{theorem}
  \begin{proof}
    Notar que \(M_{i_0} \subseteq E_{1 1} R + \cdots + E_{i_0 i_0} R\).
    Luego \(\Fila_i A = 0\) \((\forall i \in \{i_0 + 1, \dots, n\})\) para toda \(A \in M_{i_0}\).

    Notar que se \(A \in M_{i_0}\) con \(A = (E_{1 1} + \cdots + E_{i_0 i_0}) r\) para un \(r \in R\), entonces \(E_{i_0 i_0} A = E_{i_0 i_0} r\).
    En efecto
    \begin{align}
      E_{i_0 i_0} A
      &=
      E_{i_0 i_0} (E_{1 1} + \cdots + E_{i_0 i_0}) r
      \\
      &=
      (E_{i_0 i_0} E_{1 1} + \cdots + E_{i_0 i_0} E_{i_0 - 1, i_0 - 1} + E_{i_0 i_0}^2) r
      \\
      &=
      (0 + \cdots + 0 + E_{i_0 i_0}) r
      \\
      &=
      E_{i_0 i_0} r
    \end{align}
    Por este motivo \(A + M_{i_0 - 1} = A E_{i_0 i_0} + M_{i_0 - 1}\).
    Calculamos
    \begin{align}
      A + M_{i_0 - 1}
      &=
      (E_{1 1} + \cdots + E_{i_0 i_0}) r + M_{i_0 - 1}
      \\
      &=
      (E_{1 1} + \cdots + E_{i_0 - 1, i_0 - 1}) r + E_{i_0 i_0} r + M_{i_0 - 1}
      \\
      &=
      E_{i_0 i_0} r + M_{i_0 - 1}
      \\
      &=
      E_{i_0 i_0} A + M_{i_0 - 1}
    \end{align}

    Supongamos que \(A + M_{i_0 - 1} = B + M_{i_0 -1}\).
    Entonces \(E_{i_0 i_0} A + M_{i_0 - 1} = E_{i_0 i_0} B + M_{i_0 - 1}\).
    Escribamos \(C = E_{i_0 i_0} (A - B)\).
    Por un lado \(C \in M_{i_0 - 1}\) y por el otro \(C \in E_{i_0 i_0} R\).
    El primer dato implica \(\Fila_i C = 0\) para \(i \in \{i_0, \dots, n\}\).
    El segundo dato implica \(\Fila_i C = 0\) para \(i \in \{1, \dots, n\} \setminus i_0\).
    Luego \(C = 0\).
    Es decir \(E_{i_0 i_0} A = E_{i_0 i_0} B\).

    Esto nos permite definir una función \(\phi : M_{i_0} / M_{i_0 - 1} \rightarrow E_{i_0 i_0} R\) dada por \(A + M_{i_0 - 1} \mapsto E_{i_0 i_0} A\).
    Así definida, es un homomorfismo de \(R\)-módulos.

    Es además un monomorfismo.
    Si \(\phi(A + M_{i_0 - 1}) = 0\) entonces \(E_{i_0 i_0} A = 0\) y \(\Fila_{i_0} A = 0\).
    Además \(A \in M_{i_0}\) implica \(\Fila_i A = 0\) \((\forall i \in \{i_0, \dots, n\})\), y \(A \in M_{i_0 - 1}\).
    Entonces \(A + M_{i_0 - 1} = 0 + M_{i_0 - 1}\).

    También es un epimorfismo.
    Dado \(E_{i_0 i_0} A \in E_{i_0 i_0} R\), tenemos \(E_{i_0 i_0} A \in M_i\).
    Calculamos \(\phi(E_{i_0 i_0} A + M_{i_0 - 1}) = E_{i_0 i_0} (E_{i_0 i_0} A) = E_{i_0 i_0}^2 A = E_{i_0 i_0} A\).
    Por lo tanto \(E_{i_0 i_0} A \in \image \phi\).
    Concluímos que \(\phi : M_{i_0} / M_{i_0 - 1} \rightarrow E_{i_0 i_0} R\) es un isomorfismo.
  \end{proof}

  \begin{theorem}
    Sea \(R\) un anillo con identidad y \(S\) el anillo formado por todas las matrices \(n \times n\) sobre \(R\).
    \(J\) es un ideal de \(S\) si y solo si \(J\) es el anillo formado por todas las matrices \(n \times n\) sobre \(I\) para algún ideal \(I\) en \(R\).
  \end{theorem}
  \begin{proof}
    Sea \(J\) un ideal de \(S\).
    Sea \(I\) el conjunto formado por todos los elementos de \(R\) que aparecen como entrada \((1, 1)\) de alguna matriz en \(J\).
    Si \(a E \in J\) donde \(a \in R\) y \(E = E_{1 1} \in S\), entonces \(a \in I\).
    La afirmación recíproca también es verdadera.
    Notar que si \(a \in I\), entonces existe \(A = (a_{i j})\) en \(J\) con \(a_{1 1} = a\).
    Al ser \(J\) un ideal (bilátero), tenemos \(E A E \in J\).
    Pero \(E A E = a E\).
    Entonces \(a E \in J\).
    Hemos probado que \(a \in I\) si y solo si \(a E \in J\).

    Afirmamos que \(I\) es un ideal.
    En efecto, \(0 \in J\) porque \(J\) es un ideal.
    Luego \(0 \in I\) por definición de \(I\).
    Por otra parte, si \(a, b \in I\), entonces \(a E, b E \in J\).
    Pero \(J\) es un ideal.
    Entonces \((a + b) E = a E + b E \in J\).
    Luego \(a + b \in I\).
    Para finalizar consideramos \(r \in R\) y \(a \in I\).
    Entonces \(r E \in S\) y \(a E \in J\).
    Pero \(J\) es un ideal.
    Entonces
    \begin{align}
      (r a) E
      &=
      (r a) E^2
      =
      (r E) (a E)
      \in J
      \\
      (a r) E
      &=
      (a r) E^2
      =
      (a E) (r E)
      \in J
    \end{align}
    Luego \(ra, ar \in I\).

    Afirmamos que \(M_n(I) = J\).
    Sea \(A = (a_{i j})\) una matriz en \(S\).
    Comenzamos suponiendo \(A \in J\).
    Consideremos \(i, j \in \{1, \dots, n\}\).
    Porque \(J\) es un ideal, \(a_{r s} E = E_{1 r} A E_{s 1} \in J\).
    Luego \(a_{r s} \in I\).
    Porque \(i, j\) eran arbitrarios, se deduce \(A \in M_n(I)\).
    Recíprocamente, suponemos que \(A = (a_{i j}) \in M_n(I)\).
    Consideramos \(i, j \in \{1, \dots, n\}\).
    Por hipótesis \(a_{i j} \in I\).
    Luego \(a_{i j} E \in J\).
    Porque \(J\) es un ideal, se deduce \(E_{i 1} (a_{i j} E) E_{1 j} \in J\) mientras \(E_{i 1} (a_{i j} E) E_{1 j} = a_{i j} E_{i j}\).
    Porque \(i, j\) eran arbitrarios, usando que \(J\) está cerrado bajo suma, se deduce \(A = \sum_{i j} a_{i j} E_{i j} \in J\).
  \end{proof}

  \begin{theorem}
    \label{theorem:idealsInTheRingOfMatricesOverADivisionRing}
    Sea \(S\) el anillo formado por todas las matrices sobre un anillo de división \(D\).
    \begin{enumerate}
      \item
        \label{theorem:matrixRingOverDivisionRingHasNoProperIdeals}
        \(S\) no tiene ideales propios (es decir, 0 es un ideal maximal).
      \item \label{theorem:matrixRingOverDivisionRingHasZeroDivisors} \(S\) tiene divisores de cero.
      Consecuentemente,
      \begin{enumerate}
        \item \(S \simeq S / 0\) no es un anillo de división y
        \item 0 es un ideal primo a pesar de no satisfacer la condición \(a b \in I \rightarrow {a \in I} \text{ o } {b \in I}\) \((\forall a, b \in S)\)
      \end{enumerate}
    \end{enumerate}
  \end{theorem}
  \begin{proof} \ref{theorem:matrixRingOverDivisionRingHasNoProperIdeals}.
    Si \(J\) es un ideal de \(S\), entonces \(J\) es el anillo formado por todas las matrices \(n \times n\) sobre \(I\) para algún ideal \(I\) en \(D\).
    Pero \(D\) es un anillo de división, no tiene ideales propios.
    Luego \(I = 0\) o \(I = D\), concluyendo que \(J = 0\) o \(J = S\).
  \end{proof}
  \begin{proof} \ref{theorem:matrixRingOverDivisionRingHasZeroDivisors}
    Para encontrar divisores de cero basta observar la fórmula \(E_{r_1 s_1} E_{r_2 s_2} = \delta_{r_1 r_2} \delta_{s_1 s_2} E_{r_1 r_2}\).
  \end{proof}


  \begin{definition}
    Un módulo (izquierdo) \(A\) sobre un anillo \(R\) es \emph{simple} (o \emph{irreducible}) si \(R A \neq 0\) y \(A\) no tiene submódulos propios.
    Un anillo \(R\) es \emph{simple} si \(R^2 \neq 0\) y \(R\) no tiene ideales (bilaterales) propios.
  \end{definition}

  \begin{proposition}
    Todo módulo simple \(A\) es cíclico; de hecho, \(A = R a\) para todo \(a \in A\) nonulo.
  \end{proposition}
  \begin{proof}
    Ambos \(Ra\) (con \(a \in A\) nonulo) y \(B = \{c \in A : R c = 0\}\) son submódulos de \(A\), de aquí que por simplicidad cada uno de ellos sea igual a 0 o \(A\).
    También por simplicidad \(R A \neq 0\), esto implica \(B \neq A\) y \(B = 0\).
    Luego \(a \notin B\) y \(R a \neq 0\).
    En conclusión \(R a = A\).
  \end{proof}

  \begin{theorem}
    \label{theorem:anihilatorTheorem}
    Sea \(B\) un subconjunto de un módulo izquierdo sobre un anillo \(R\).
    Entonces \(\mathcal{A}(B) = \{r \in R \mid r b = 0 (\forall b \in B)\}\) es un ideal izquierdo de \(R\).
    Si \(B\) es un submódulo de \(A\), entonces \(\mathcal{A}(B)\) es un ideal.
  \end{theorem}
  \(\mathcal{A}(B)\) es el \emph{aniquilador (izquierdo)} de \(B\).
  El aniquilador derecho de un módulo derecho se define análogamente.

  \begin{definition}
    Un módulo (izquierdo) \(A\) es \emph{fiel} si su aniquilador (izquierdo) \(\mathcal{A}(A)\) es 0.
    Un anillo \(R\) es \emph{primitivo} (\emph{izquierdo}) si existe un \(R\)-módulo simple y fiel.
  \end{definition}

  Los anillos primitivos derechos se definen análogamente.
  Sí existen anillos primitivos derechos que no son primitivos izquierdos.
  De aquí en más \emph{primitivo} siempre significará \emph{primitivo izquierdo}.
  Sin embargo, todos los resultados probados para anillos primitivos izquierdos son verdaderos, mutatis mutandis, para anillos primitivos derechos.

  \begin{definition}
    Sea \(V\) un espacio vectorial izquierdo sobre un anillo de división \(D\).
    Un subanillo \(R\) del anillo de endomorfismos \(\Hom_D(V, V)\) es un \emph{anillo denso de endomorfismos} de \(V\) (o un \emph{subanillo denso} de \(\Hom_D(V, V)\)) si para todo entero positivo \(n\), cada subconjunto linealmente independiente \(\{u_1, \dots, u_n\}\) de \(V\) y cada subconjunto arbitrario \(\{v_1, \dots, v_n\}\) de \(V\), existe \(\theta \in R\) tal que \(\theta(u_i) = v_i\) \((\forall i \in \{1, \dots, n\})\).
  \end{definition}

  \begin{lemma}
    Sea \(A\) un módulo simple sobre un anillo \(R\).
    Consideramos \(A\) como un espacio vectorial sobre el anillo de división \(D = \Hom_R(A, A)\).
    Si \(V\) es un subespacio finito-dimensional del \(D\)-espacio vectorial \(A\) y \(a \in A \setminus V\), entonces existe \(r \in R\) tal que \(r a \neq 0\) y \(r V = 0\).
  \end{lemma}
  \begin{proof}
    La prueba es por inducción sobre \(n = \dim_D V\).
    Comenzamos por el caso base.
    Si \(n = 0\), entonces \(V = 0\) y \(a \neq 0\).
    Porque \(A\) es simple, \(a \neq 0\) implica \(R a = A\).
    Consecuentemente existe \(r \in R\) tal que \(r a = a \neq 0\) y \(r V = r 0 = 0\).

    En el paso inductivo, supongamos \(\dim_D V = n > 0\) y que el teorema es verdadero para dimensiones menores a \(n\).
    Sea \(\{u_1, \dots, u_{n - 1}, u\}\) una \(D\)-base de \(V\) y sea \(W\) el subespacio \((n - 1)\)-dimensional generado por \(\{u_1, \dots, u_{n - 1}\}\) (siendo \(W = 0\) cuando \(n = 1\)).
    Entonces \(V = W \oplus D u\) (suma directa de espacios vectoriales).
    Nuestra hipótesis inductiva tiene dos consecuencias importantes:
    \begin{enumerate}
      \item para todo \(v \in A \setminus W\) existe \(r \in R\) tal que \(r u \neq 0\) y \(r W = 0\);
      \item para todo \(v \in A\), si \(r v = 0\) para todo \(r \in R\) entonces \(v \in W\).
    \end{enumerate}
    La primera consecuencia implica que existe \(r \in R\) tal que \(r u \neq 0\) y \(r W = 0\).
    Pero \(r W = 0\) si y solo si \(r \in \mathcal{A}(W)\), siendo \(I = \mathcal{A}(W)\) un ideal izquierdo de \(R\).
    Además \(r u \in I u \setminus 0\), siendo \(I u\) un submódulo de \(A\).
    Por simplicidad, este submódulo nonulo debe ser \(I u = A\).

    Para terminar el argumento inductivo, debemos encontrar \(r \in R\) tal que \(r a \neq 0\) y \(r V = 0\).
    Si no existe tal \(r\), entonces podemos definir una aplicación \(\theta : A \rightarrow A\) como sigue.
    Para \(r u \in I u = A\) definimos \(\theta(r u) = r a \in A\).
    Afirmamos que \(\theta\) está bien definida.
    Sean \(r_1, r_2 \in I\) tales que \(r_1 u = r_2 u\).
    Por hipótesis \((r_1 - r_2) a = 0\) o \((r_1 - r_2) V \neq 0\).
    Ahora bien, porque \(r_1 - r_2 \in I = \mathcal{A}(W)\) tenemos \((r_1 - r_2) W = 0\);
    y porque \(D = \Hom_D(A, A)\), para cada \(d \in D\) tenemos \((r_1 - r_2) (d \cdot u) = (r_1 - r_2) d(u) = d((r_1 - r_2) u) = d(0) = 0\).
    Juntos, estos dos datos implican \((r_1 - r_2) V = (r_1 - r_2) (W \oplus D u) = 0\).
    Consecuentemente, por hipótesis \((r_1 - r_2) a = 0\).
    Por lo tanto \(\theta(r_1 u) = r_1 a = r_2 a = \theta(r_2 u)\).
    Podemos mostrar que \(\theta \in \Hom_D(A, A) = D\).
    Luego para cada \(r \in I\), \(0 = \theta(r u) - r a = r \theta(u) - r a = r(\theta(u) - a)\).
    De aquí que \(\theta(u) - a \in W\), por la segunda consecuencia de la hipótesis inductiva.
    Consecuentemente \(a = \theta u - (\theta u - a) \in D u + W = V\),
    lo cual contradice el hecho \(a \notin V\).
    Por lo tanto, existe \(r \in R\) tal que \(r a \neq 0\) y \(r V = 0\).
  \end{proof}

  \begin{theorem}[de Densidad de Jacobson]
    Sea \(R\) un anillo primitivo y \(A\) un \(R\)--módulo simple y fiel.
    Considerar \(A\) como espacio vectorial sobre el anillo de división \(\Hom_R(A, A) = D\).
    Entonces \(R\) es isomorfo a un anillo denso de endomorfismos de \(D\)--espacio vectorial \(A\).
  \end{theorem}
  \begin{proof}
    Para cada \(r \in R\) la aplicación \(\alpha_r : A \rightarrow A\) dada por \(\alpha_r(a) = r a\) es facilmente identificada como un \(D\)--endomorfismo de \(A\): esto es, \(\alpha_r \in \Hom_D(A, A)\).
    Además para todo par \(r, s \in R\) se verifican \(\alpha_{(r + s)} = \alpha_r + \alpha_s\) y \(\alpha_{r s} = \alpha_r \alpha_s\).
    Consecuentemente la aplicación \(\alpha : R \rightarrow \Hom_D(A, A)\) definida por \(\alpha(r) = \alpha_r\) es un homomorfismo de anillos bien definido.
    Dado que \(A\) es un \(R\)--módulo fiel, \(\alpha_r = 0\) si y solo si \(r \in \mathcal{A}(A) = 0\).
    De aquí que \(\alpha\) es un monomorfismo, y \(R\) es isomorfo al subanillo \(\image \alpha\) de \(\Hom_D(A, A)\).

    Para completar la prueba debemos mostrar que \(\image \alpha\) es un subanillo denso de \(\Hom_D(A, A)\).
    Dado un subconjunto \(D\)--linealmente independiente \(\{u_1, \dots, u_n\}\) de \(A\), y un subconjunto arbitrario \(\{v_1, \dots, v_n\}\) de \(A\), debemos encontrar \(\alpha_r \in \image \alpha\) tal que \(\alpha_r(u_i) = v_i\) \((\forall i \in \{1, \dots, n\})\).
    Para cada \(i\) sea \(V_i\) el \(D\)--subespacio de \(A\) generado por \(\{u_j : j \neq i\}\).
    Dado que \(\{u_1, \dots, u_n\}\) es linealmente independiente, \(u_i \notin V_i\).
    Consecuentemente, por el lema \ref{} existe \(r_i \in R\) tal que \(r_i u_i \neq 0\) y \(r_i V_i = 0\).
    Después aplicamos el lema \ref{} al subespacio nulo y a elemento nonulo \(r_i u_i\):
    existe \(s_i \in R\) tal que \(s_i r_i u_i \neq 0\) y \(s_i 0 = 0\).
    Siendo \(s_i r_i u_i \neq 0\), el \(R\) submódulo \(R (r_i u_i)\) de \(A\) es nonulo, luego \(R(r_i u_i) = A\) por simplicidad.
    Por esto existe \(t_i \in R\) tal que \(t_i r_i u_i = v_i\).
    Sea \(r = t_1 r_1 + t_2 r_2 + \cdots + t_n r_n\).
    Recordar que \(u_i \in V_j\) para \(i \neq j\), luego \(t_j r_j u_i \in t_j (r_j V_i) = t_j 0 = 0\).
    Consecuentemente \(\alpha_r(u_i) = (t_1 r_1 + \cdots + t_n r_n) u_i = r_i r_i u_i = v_i\).
    Por lo tanto \(\image \alpha\) es un anillo denso de endomorfismos de \(D\)--espacio vectorial \(A\).
  \end{proof}

  \begin{definition}
    Decimos que un módulo \(A\) satisface la \emph{condición de la cadena ascendente (ACC) sobre submódulos} (o decimos que es \emph{noetheriano}) si para toda cadena \(A_1 \subseteq A_2 \subseteq A_3 \subseteq \cdots\) de submódulos de \(A\), existe un entero \(m\) tal que \(B_i = B_m\) para todo \(i \geq m\).
  \end{definition}

  Si un anillo \(R\) es pensado como módulo izquierdo (resp. derecho) sobre si mismo, entonces es facil ver que los submódulos de \(R\) son precisamente los ideales izquierdos (resp. derechos) de \(R\).
  Consecuentemente, en este caso se acostumbra hablar de condiciones de cadena sobre ideales (izquierdos o derechos) en lugar de submódulos.

  \begin{definition}
    Un anillo \(R\) es \emph{noetheriano izquierdo} (resp. \emph{derecho}) si \(R\) satisface la condición de la cadena ascendente sobre ss ideales izquierdos (resp. derechos).
    Se dice que \(R\) es \emph{noetheriano} si \(R\) es noetheriano izquierdo y derecho a la vez.

    Un anillo \(R\) es \emph{artiniano izquierdo} (resp. \emph{derecho}) si \(R\) satisface la condición de la cadena descendiente sobre sus ideales izquierdos (resp. derechos).
    Se dice que \(R\) es artiniano si \(R\) es artiniano izquierdo y derecho a la vez.
  \end{definition}

  \begin{definition}
    Un módulo \(A\) satisface la \emph{condición maximal} [resp. \emph{minimal}] \emph{sobre submódulos} si todo conjunto novacío de submódulos de \(A\) contiene un elemento maximal [resp. minimal] (con respecto al orden dado por la inclusión de conjuntos).
  \end{definition}

  \begin{theorem}
    \label{theorem:equivalenceOfChainConditions}
    Un módulo satisface la condición de la cadena ascendente [resp. descencendente] sobre submódulos si y solo si satisface la condición maximal [resp. minimal] sobre submódulos.
  \end{theorem}
  \begin{proof}
    Supongamos que el módulo \(A\) satisface la condición minimal sobre submódulos y que \(A_1 \supseteq A_2 \supseteq \cdots\) es una cadena de submódulos.
    Entonces el conjunto \(\{A_i \mid i \geq 1\}\) tiene un elemento minimal, digamos \(A_n\).
    Consecuentemente, para \(i \geq n\) tenemos \(A_n \supseteq A_i\) por hipótesis y \(A_n \subseteq A_i\) por minimalidad, luego \(A_i = A_n\) para todo \(i \geq n\).
    Por lo tanto, \(A\) satisface la condición descendiente de la cadena.

    Recíprocamente supongamos que \(A\) satisface la condición de la cadena descendente, y \(S\) es un conjunto novacío de submódulos de \(A\).
    Entonces existe \(B_0 \in S\).
    Si \(S\) no tiene elemento minimal, entonces para todo submódulo \(B\) en \(S\) existe al menos un submódulo \(B'\) en \(S\) tal que \(B \supset B'\).
    Para cada \(B\) en \(S\), elegimos uno de estos \(B'\) (Axioma de Elección).
    Esta elección define una función \(f : S \rightarrow S\) mediante \(B \mapsto B'\).
    Por el Teorema de la Recursión, existe una función \(\phi : \naturalNumbers \rightarrow S\) tal que \(\phi(0) = B_0\) y \(\phi(n + 1) = f(\phi(n))\) \((\forall n \in \naturalNumbers)\).
    Por tanto si \(B_n = \phi(n)\) \((\forall n \in \naturalNumbers)\), entonces \(B_0 \supset B_1 \supset \cdots\) es una cadena descendiente que viola la condición descendiente de la cadena.
    Por lo tanto, \(S\) debe tener un elemento minimal.
    Concluímos que \(A\) satisface la condición minimal.

    La prueba para las condiciones de la cadena ascendente y maximal es análoga.
  \end{proof}

  \begin{theorem}
    \label{theorem:leftArtinianDenseEndomorphismRingsOfVectorSpaces}
    Sea \(R\) un anillo denso de endomorfismos de un espacio vectorial \(V\) sobre un anillo de división \(D\).
    Entonces \(R\) es artiniano izquierdo [resp. derecho] si y solo si \(\dim_D V\) es finita, en cuyo caso \(R = \Hom_D(V, V)\).
  \end{theorem}
  \begin{proof}
    Si \(R\) es artiniano izquierdo, y \(\dim_D V\) es infinita, entonces existe un subconjunto de \(V\) linealmente independiente e infinito (numerable) \(\{u_1, u_2, \dots\}\).
    Por el Ejercicio IV.1.7 \(V\) es un \(\Hom_D(V, V)\)-módulo izquierdo y por tanto un \(R\)-módulo izquierdo (recordar que \(R \subseteq \Hom_D(V, V)\)).
    Para cada \(n\) sea \(I_n\) el aniquilador izquierdo en \(R\) del conjunto \(\{u_1, \dots, u_n\}\).
    Por el Teorema \ref{theorem:anihilatorTheorem} \(I_1 \supseteq I_2 \supseteq \cdots\) es una cadena descendente de ideales izquierdos de \(R\).
    Sea \(w\) un elemento nonulo de \(V\), no importa cual de ellos sea (podría ser \(u_1\), por ejemplo).
    Dado que \(\{u_1, \dots, u_{n + 1}\}\) es linealmente independiente (para cada \(n\)) y \(R\) es denso, existe \(\theta \in R\) tal que \(\theta u_i = 0\) \((\forall i \in \{1, \dots, n\})\) y \(\theta u_{n + 1} = w \neq 0\).
    Consecuentemente \(\theta \in I_n\) pero \(\theta \notin I_{n + 1}\).
    Por lo tanto \(I \supset I_2 \supset \cdots\) es una cadena estrictamente descencendente, su existencia lleva a una contradicción.
    Luego \(\dim_D V\) es finita.

    Recíprocamente, si \(\dim_D V\) es finita, entonces \(V\) tiene una base finita \(\{v_1, \dots, v_m\}\).
    Si \(f\) es un elemento de \(\Hom_D(V, V)\), entonces \(f\) está completamente determinado por su acción sobre \(v_1, \dots, v_m\) por los teoremas \ref{theorem:freeUnitalModulesOverARingWithIdentity} y \ref{theorem:vectorSpaceBasis}.
    Dado que \(R\) es denso, existe \(\theta \in R\) tal que \(\theta v_i = f v_i\) \(\forall i \in \{1, \dots, m\}\).
    Luego \(f = \theta \in R\).
    Por lo tanto \(\Hom_D(V, V) = R\).
    Pero \(\Hom_D(V, V)\) es artiniano por el Teorema \ref{theorem:isomorphismBetweenMatrixAndHomomorphismRings} y el corolario VIII.1.12.
  \end{proof}

  \begin{theorem}[de Densidad de Jacobson]
    \label{theorem:jacobsonDensityTheorem}
    Sea \(R\) un anillo primitivo y \(A\) un \(R\)-módulo simple y fiel.
    Considerar \(A\) como espacio vectorial sobre el anillo de división \(\Hom_R(A, A) = D\).
    Entonces \(R\) es isomorfo a un anillo denso de endomorfismos del \(D\)-espacio vectorial \(A\).
  \end{theorem}
  \begin{proof}
    Para cada \(r \in R\) la aplicación \(\alpha_r : A \rightarrow A\) dada por \(\alpha_r(a) = r a\) es facilmente identificada como un \(D\)-endomorfismo de \(A\): esto es, \(\alpha_r \in \Hom_D(A, A)\).
    Además \(\alpha_{(r + s)} = \alpha_r + \alpha_s\) y \(\alpha_{r s} = \alpha_r \alpha_s\) para todo par \(r, s \in R\).
    Consecuentemente la aplicación \(\alpha : R \rightarrow \Hom_D(A, A)\) definida por \(\alpha(r) = \alpha r\) es un homomorfismo de anillos bien definido.
    Dado que \(A\) es un \(R\)-módulo fiel, \(\alpha_r = 0\) si y solo si \(r \in \mathcal{A}(A) = 0\).
    De aquí que \(\alpha\) es un monomorfismo, y \(R\) es isomorfo al subanillo \(\image \alpha\) de \(\Hom_D(A, A)\).

    Para completar la prueba debemos mostrar que \(\image \alpha\) es un subanillo denso de \(\Hom_D(A, A)\).
    Sea \(U = \{u_1, \dots, u_n\}\) un subconjunto \(D\)-linealmente independiente de \(A\); y sea \(\{v_1, \dots, v_n\}\) un subconjunto arbitrario de \(A\).
    Debemos encontrar \(\alpha_r \in \image \alpha\) tal que \(\alpha_r(u_i) = v_i\) \((\forall i \in \{1, \dots, n\})\).
    Para cada \(i\) sea \(V_i\) el \(D\)-subespacio de \(A\) generado por \(\{u_j : j \neq i\}\).
    Dado que \(U\) es linealmente independiente, \(u_i \notin V_i\).
    Consecuentemente, por el lema 1.11 existe \(r_i \in R\) tal que \(r_i u_i \neq 0\) y \(r_i V_i = 0\).
    Después aplicamos el lema 1.11 al subespacio nulo y al elemento nonulo \(r_i u_i\):
    existe \(s_i \in R\) tal que \(s_i r_i u_i \neq 0\) y \(s_i 0 = 0\).
    Siendo \(s_i r_i u_i \neq 0\), el \(R\)-submódulo \(R (r_i u_i)\) de \(A\) en nonulo, luego \(R (r_i u_i) = A\) por simplicidad.
    Por esto existe \(t_i \in R\) tal que \(t_i r_i u_i = v_i\).
    Sea \(r = t_1 r_1 + t_2 r_2 + \cdots + t_n r_n \in R\).
    Recordar que \(u_i \in V_j\) para \(i \neq j\), luego \(t_j r_j u_i \in t_j (r_j V_i) = t_j 0 = 0\).
    Consecuentemente \(\alpha_r(u_i) = (t_1 r_1 + \cdots + t_n r_n) u_i = t_i r_i u_i = v_i\).
    Por lo tanto \(\image \alpha\) es un anillo denso de endomorfismos del \(D\)-espacio vectorial \(A\).
  \end{proof}

  \begin{definition}
    Una \emph{serie subnormal} de un grupo \(G\) es una cadena de subgrupos \(G = G_0 \geq G_1 \geq \cdots \geq G_n = \langle e \rangle\) tal que \(G_{i + 1}\) es normal en \(G_i\) para \(1 \leq i \leq n\).
    Los \emph{factores} de la serie son los grupos cociente \(G_i / G_{i + 1}\).
    La \emph{longitud} de la serie es el número de inclusiones estrictas (alternativamente, el número de factores con orden mayor a 1).
    Una serie subnormal es una \emph{serie de composición} si cada factor \(G_i / G_{i + 1}\) es simple.
  \end{definition}

  Una \emph{serie normal} para un módulo \(A\) es una cadena de submódulos:
  \(A = A_0 \supseteq A_1 \supseteq A_2 \supseteq \cdots \supseteq A_n\).
  Los \emph{factores} de la serie son los módulos cociente \(A_i / A_{i + 1}\) \((0 \leq i < n)\).
  La \emph{longitud} de la serie es el número de inclusiones propias (igual al número de factores notriviales).
  Uni \emph{refinamiento propio} es un refinamiento con longitud mayor a la serie original.
  Dos series normales son \emph{equivalentes} si existe una correspondencia uno-a-uno entre los factores notriviales tal que factores correspondientes sean isomorfos.
  De tal modo, series equivalentes tienen igual longitud.
  Una \emph{serie de composición} para \(A\) es una serie normal
  \(A = A_0 \supseteq A_1 \supseteq A_2 \supseteq \cdots \supseteq A_n = 0\)
  tal que cada factor \(A_k / A_{k + 1}\) \((0 \leq k < n)\) es un módulo nonulo sin submódulos propios.
  Si \(R\) es unitario, decimos que un módulo unitario sin submódulos propios es \emph{simple}.

  La Teoría de Series Normales y Subnormales para grupos puede trasladarse al caso de los módulos.
  Como consecuencia de esta tenemos el siguiente teorema.
  \begin{theorem}
    \label{theorem:refinementAndEquivalenceOfNormalSeries}
    Cualesquiera dos series normales de un módulo \(A\) tienen refinamientos que son equivalentes.
    Caulesquiera dos series de composición de \(A\) son equivalentes.
  \end{theorem}

  \begin{theorem}
    Un módulo nonulo \(A\) tiene una serie de composición is y solo si \(A\) satisface tanto la condición de la cadena descencendente como la ascendente.
  \end{theorem}
  \begin{proof}
    Supongamos que \(A\) tiene una serie de composición \(S\) de longitud \(n\).
    Si alguna de las condiciones de la cadena falla, podemos encontrar submódulos \(A = A_0 \supset A_1 \supset A_2 \supset \cdots \supset A_n \supset A_{n + 1}\)
    que forman una serie normal \(T\) de longitud \(n + 1\).
    Por el teorema \ref{theorem:refinementAndEquivalenceOfNormalSeries}, \(S\) y \(T\) tienen refinamientos equivalentes.
    Esto es una contradicción porque series equivalentes tienen igual longitud.
    Todo refinamiento de la series de composición \(S\) tiene longitud \(n\) al igual que \(S\), pero todo refinamiento de \(T\) tiene longitud al menos \(n + 1\).
    Por lo tanto \(A\) satisface ambas condiciones de la cadena.

    Recíprocamente, suponemos que \(B\) es un submódulo nonulo de \(A\), definimos \(S(B)\) como el conjunto formado por todos los submódulos \(C\) de \(B\) con \(C \neq B\).
    De tal modo que si \(B\) no tiene submódulos propios, entonces \(S(B) = \{0\}\).
    También definimos \(S(0) = \{0\}\).
    Para cada \(B\), el conjunto \(S(B)\) tiene un elemento maximal \(B'\) (por el Teorema \ref{theorem:equivalenceOfChainConditions}
    ).
    Sea \(S\) el conjunto de todos los submódulos de \(A\).
    Definimos una aplicación \(f : S \rightarrow S\) mediante \(f(B) = B'\) (usando el Axioma de Elección).
    Por el Teorema de la Recursión, existe una función \(\phi : \naturalNumbers \rightarrow S\) tal que \(\phi(0) = a\) y \(\phi(n + 1) = f(\phi(n))\).
    Si \(A_i = \phi(i)\), entonces \(A \supseteq A_1 \supseteq A_2 \supseteq \cdots\) es una cadena descendiente por construcción.
    Luego para un \(n\), \(A_i = A_n\) \((\forall i \geq n)\).
    Dado que \(A_{n + 1} = f(A_n)\), la definición de \(f\) muestra que \(A_{n + 1} = A_n\) solo si \(A_n = 0 = A_{n + 1}\).
    Sea \(m\) el menor entero tal que \(A_m = 0\).
    Entonces \(m \leq n\) y \(A_k \neq 0\) \((\forall k < m)\).
    Más aún, para cada \(k < m\), \(A_{k + 1}\) es un submódulo maximal de \(A_k\) tal que \(A_k \supseteq A_{k + 1}\).
    Consecuentemente, cada \(A_k / A_{k + 1}\) es nonulo y no tiene submódulos propios por el \ref{theorem:submodulesOfQuotientModule}.
    Por lo tanto \(A \supseteq A_1 \supseteq \cdots \supseteq A_m = 0\) es una serie de composición para \(A\).
  \end{proof}

  \begin{theorem}[de Artin--Wedderburn]
    Las siguientes condiciones sobre un anillo artiniano izquierdo \(R\) son equivalentes.
    \begin{enumerate}
      \item
        \label{property:artin--wedderburn_simple-ring}
        \(R\) es simple;
      \item
        \label{property:artin--wedderburn_primitive-ring}
        \(R\) es primitivo;
      \item
        \label{property:artin--wedderburn_endomorphism-ring}
        \(R\) es isomorfo al anillo de endomorfismos de un espacio vectorial nonulo sobre un anillo de división \(D\);
      \item
        \label{property:artin--wedderburn_matrix-ring}
        para algún entero positivo \(n\), \(R\) es isomorfo al anillo formado por las matrices \(n \times n\) sobre un anillo de división.
    \end{enumerate}
  \end{theorem}
  \begin{proof}
    \(\ref{property:artin--wedderburn_simple-ring} \Rightarrow \ref{property:artin--wedderburn_primitive-ring}\).
    Primero observamos que \(I = \{r \in R \mid R r = 0\}\) es un ideal de \(R\), con la propiedad \(I R = 0\).
    Pero \(R\) es simple: no tiene ideales propios, por lo cual \(I = R\) o \(I = 0\);
    y \(R R \neq 0\), por lo cual \(I = 0\).

    Consideremos el conjunto \(\mathcal{S}\) formado por todos los ideales izquierdos nonulos de \(R\).
    Dado que \(R\) es artiniano izquierdo, satisface la condición de la cadena descendiente sobre ideales izquierdos.
    En particular, para toda sucesión \(\{S_i\}_{i \in \naturalNumbers}\) en \(\mathcal{S}\) con \(S_0 \supseteq S_1 \supseteq S_2 \supseteq \cdots\), existe un \(m \in \naturalNumbers\) tal que \(S_m = S_i\) para todo \(i \geq m\).
    El Lema de Zorn permite deducir de esto la existencia de un elemento minimal \(J \in \mathcal{S}\), tal que \(J \supseteq J' \rightarrow J = J'\) para todo \(J' \in \mathcal{S}\).
    Esta minimalidad hace que \(J\) no tenga \(R\)--submódulos propios (un \(R\)--submódulo de \(J\) es un ideal izquierdo de \(R\) contenido en \(J\)).

    Afirmamos que el aniquilador izquierdo \(\mathcal{A}(J)\) de \(J\) en \(R\) es cero.
    De otro modo \(\mathcal{A}(J) = R\) por simplicidad y \(R u = 0\) para cada \(u \in J\) nonulo.
    Consecuentemente, cada uno de estos \(u\) nonulos pertenece a \(I = 0\), lo cual es una contradicción.
    Por lo tanto \(\mathcal{A}(J) = 0\) y \(R J \neq 0\).
    En conclusión, \(J\) es un \(R\)--módulo simple y fiel, y \(R\) es primitivo.

    \(\ref{property:artin--wedderburn_primitive-ring} \Rightarrow \ref{property:artin--wedderburn_endomorphism-ring}\)
    Por el Teorema de Densidad de Jacobson \ref{theorem:jacobsonDensityTheorem}, \(R\) es isomorfo a un anillo denso \(T\) compuesto por endomorfismos de un espacio vectorial \(V\) sobre un anillo de división \(D\).
    Porque \(R\) es artiniano izquierdo, \(R \simeq T = \Hom _D(V, V)\) por el Teorema \ref{theorem:leftArtinianDenseEndomorphismRingsOfVectorSpaces}.

    \(\ref{property:artin--wedderburn_endomorphism-ring} \Leftrightarrow \ref{property:artin--wedderburn_matrix-ring}\)
    Teorema \ref{theorem:isomorphismBetweenMatrixAndHomomorphismRings}

    \(\ref{property:artin--wedderburn_matrix-ring} \Leftrightarrow \ref{property:artin--wedderburn_simple-ring}\)
    Teorema \ref{theorem:idealsInTheRingOfMatricesOverADivisionRing}
  \end{proof}
\end{document}