\documentclass{report}
\usepackage[spanish]{babel}
\title{Prerequisitos y preliminares \\ Hungerford}
\author{Pablo Brianese}

\usepackage{mathtools, amsthm, amssymb}
\usepackage{tikz-cd}

\newcommand{\naturalNumbers}{\mathbb{N}}
\newcommand{\integerNumbers}{\mathbb{Z}}

\DeclarePairedDelimiter{\abs}{\lvert}{\rvert}

\newtheorem{theorem}{Teorema}
\begin{document}
  \maketitle

  \begin{theorem}[4.1]
    Sea \(A\) un conjunto novacio.
    Dada una relación de equivalencia \(R \subseteq A \times A\), definimos sus clases de equivalencia como \(\bar{a} = \{b \in A : (a, b) \in R\}\) para cada \(a \in A\), y definimos el cociente de \(A\) por \(R\) como \(A / R = \{\bar{a} : a \in A\}\).
    La asignación \(R \mapsto A / R\) define una biyección entre el conjunto \(E(A)\), formado por todas las relaciones de equivalencia \(R\) sobre \(A\), y el conjunto \(Q(A)\), formado por todas las parciciones de \(A\).
  \end{theorem}

  \begin{theorem}[5.2]
    Sea \(\{A_i : i \in I\}\) una familia de conjuntos indexada por \(I\).
    Entonces existe un conjunto \(D\), junto con una familia de aplicaciones \(\{\pi_i : D \rightarrow A_i | i \in I\}\) con la siguiente propiedad:
    para cualquier conjunto \(C\) y familia de aplicaciones \(\{\phi_i : C \rightarrow A_i | i \in I\}\), existe una única aplicación \(\psi : C \rightarrow D\) tal que \(\pi_i \phi = \phi_i\) para todo \(i \in I\).
    Más aún, \(D\) queda determinado univocamente salvo una biyección.
    \[
      \begin{tikzcd}
        C \arrow{rd}{\phi_i} \arrow[dotted]{d}{\phi} \\
        D \arrow{r}{\pi_i} & A_i
      \end{tikzcd}
    \]
  \end{theorem}

  \begin{theorem}[6.2 Recursion]
    Si \(S\) es un conjunto, \(a \in S\) y para cada \(n \in \naturalNumbers\), \(f_n : S \rightarrow S\) es una función, entonces existe una única función \(\phi : \naturalNumbers \rightarrow S\) tal que \(\phi(0) = a\) y \(\phi(n + 1) = f_n(\phi(n))\) para todo \( n \in \naturalNumbers\).
  \end{theorem}

  Para cada \(N \in \integerNumbers\), denotamos \([N] = \{n \in \naturalNumbers : 0 \leq n \leq N\}\).

  \begin{proof}
    Sea \(\mathcal{N}\) el conjunto de los \(N \in \naturalNumbers\) tales que existe una única función \(\phi_N : [N] \rightarrow S\) que verifica la \emph{condición recursiva} \(\phi_N(n + 1) = f_n ( \phi_N(n))\) para todo \(n \in [N - 1]\), y la condición base \(\phi_N(0) = a\).

    En un principio \(0 \in \mathcal{N}\).
    En efecto, aquí la condición \(\phi_0(0) = a\) determina unívocamente a la función \(\phi_0 : \{0\} \rightarrow S\);
    y la condición recursiva sobre \(\phi_0\) es vacua porque \([- 1] = \emptyset\).

    Supongamos, inductivamente, que \(N \in \mathcal{N}\).
    Entonces existe una única función \(\phi_N : [N] \rightarrow S\) tal que \(\phi_N(0) = a\) y verifica la condición recursiva.
    Definimos \(\phi_{N + 1} : [N + 1] \rightarrow S\) como \(\phi_{N + 1}(n) = \phi_N(n)\) para \(n \in [N]\), e imponemos \(\phi_{N + 1}(N + 1) = f_N(\phi_N(N))\).
    Entonces por construcción \(\phi_{N + 1}(0) = a\), y \(\phi_{N + 1}\) verifica la condición recursiva.
    Para probar la unicidad de \(\phi_{N + 1}\) suponemos que \(\psi : [N + 1] \rightarrow S\) es cualquier función que verifique la condición recursiva y la condición base.
    Si restringimos \(\psi\) al conjunto \([N]\) obtenemos una función que satisface la condición recursiva y la condición base.
    Por hipótesis inductiva \(\left. \psi \right|_{[N]} = \phi_N\).
    La condición recursiva para \(\psi\) dice que \(\psi(N + 1) = f_N(\psi(N))\).
    Pero probamos \(\psi(N) = \phi_N(N)\).
    Entonces \(\psi(N + 1) = f_N(\phi_N(N))\).
    Por lo tanto \(\psi = \phi_{N + 1}\).
    En conclusión \(N + 1 \in \mathcal{N}\).

    Por inducción, para todo \(N \in \naturalNumbers\) existe una única función \(\phi_N : [N] \rightarrow S\) tal que \(\phi_N(0) = a\) y \(\phi_N(n + 1) = f_n ( \phi_N(n))\) para todo \(n \in [N - 1]\).
    La propiedad de unicidad las hace compatibles, es decir que \(\left. \phi_N \right|_{[M]} = \phi_M\) si \(M \leq N\).
    Esta compatibilidad nos permite afirmar que la unión de sus gráficas, \(\bigcup_{N \in \naturalNumbers} \text{gr}(\phi_N)\), es la gráfica de una función \(\phi : \naturalNumbers \rightarrow S\) tal que \(\phi(0) = a\) y \(\phi(n + 1) = f_n(\phi(n))\) para todo \( n \in \naturalNumbers\).
    La unicidad de esta gran \(\phi\) es consecuencia de la unicidad de las pequeñas \(\phi_N\) con \(N \in \naturalNumbers\).
  \end{proof}

  \begin{theorem}[6.3 Algoritmo de la División]
    Si \(a, b \in \integerNumbers\) y \(a \neq 0\), entonces existen enteros \(q\) y \(r\), únicos tales que \(b = a q + r\) y 
    \(0 \leq r < \abs{a}\).
  \end{theorem}
  \begin{proof}
    Siendo que \(a \neq 0\), el conjunto \(a \integerNumbers\) no está acotado inferiormente.
    Luego, existe un entero en \(a \integerNumbers\) menor o igual a \(b\).
    Por eso el conjunto \(S = (b - a \integerNumbers) \cap \naturalNumbers\) no vacío.
    En tanto es un subconjunto novacío de \(\naturalNumbers\), \(S\) contiene un elemento mínimo \(r = b - a q\) (para algún \(q \in \integerNumbers\)).

    Tenemos, en primer lugar, que \(b = a q + r\).
    También sabemos \(0 \leq r\) porque \(r \in \naturalNumbers\).

    Supongamos, para llegar a un absurdo, que \(r \geq \abs{a}\).
    Escribimos \(\abs{a} = a \sigma\) con \(\sigma = \pm 1\).
    Así, \(\sigma \in \integerNumbers\) implica que \(q' = q + \sigma\), \(r' = r - a \sigma\) son números enteros tales que \(b = a q + r = a (q + \sigma) + (r - a \sigma) = a q' + r'\).
    Más aún, \(r \geq \abs{a}\) implica \(r' \geq 0\).
    Luego \(r' \in S\), conjunto que tiene a \(r\) como elemento mínimo.
    Se sigue que \(r \leq r'\).
    Pero \(a \sigma > 0\) porque \(a \neq 0\), luego \(r' = r - a \sigma < r\).
    Esto es absurdo.
    Por lo tanto \(r < \abs{a}\).

    Para probar la unicidad de \(q\), \(r\), suponemos que \(q'\), \(r'\) son enteros tales que \(b = a q' + r'\) y \(0 \leq r' < \abs{a}\).
    Luego, las ecuaciones \(b = a q + r\) y \(b = a q' + r'\) implican \(- a (q' - q) = r' - r\); y las desigualdades \(0  \leq r < \abs{a}\) y \(0 \leq r' < \abs{a}\) implican \(- \abs{a} < r' - r < \abs{a}\).
    Juntas, permiten deducir que \(\abs{- a (q' - q)} < \abs{a}\) y \(\abs{q' - q} < 1\).
    Pero \(\abs{q' - q} \in \naturalNumbers\), y el menor número natural positivo es \(1\), por lo tanto \(\abs{q' - q} = 0\).
    Concluímos \(q' - q = 0\), y \(r' - r = - a \cdot 0 = 0\).
  \end{proof}
\end{document}