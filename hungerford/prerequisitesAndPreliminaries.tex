\documentclass{report}
\usepackage[spanish]{babel}
\title{Prerequisitos y preliminares \\ Hungerford}
\author{Pablo Brianese}

\usepackage{mathtools, amsthm, amssymb}
\usepackage{tikz-cd}

\newtheorem{theorem}{Teorema}
\begin{document}
  \maketitle

  \begin{theorem}[4.1]
    Sea \(A\) un conjunto novacio.
    Dada una relación de equivalencia \(R \subseteq A \times A\), definimos sus clases de equivalencia como \(\bar{a} = \{b \in A : (a, b) \in R\}\) para cada \(a \in A\), y definimos el cociente de \(A\) por \(R\) como \(A / R = \{\bar{a} : a \in A\}\).
    La asignación \(R \mapsto A / R\) define una biyección entre el conjunto \(E(A)\), formado por todas las relaciones de equivalencia \(R\) sobre \(A\), y el conjunto \(Q(A)\), formado por todas las parciciones de \(A\).
  \end{theorem}

  \begin{theorem}[5.2]
    Sea \(\{A_i : i \in I\}\) una familia de conjuntos indexada por \(I\).
    Entonces existe un conjunto \(D\), junto con una familia de aplicaciones \(\{\pi_i : D \rightarrow A_i | i \in I\}\) con la siguiente propiedad:
    para cualquier conjunto \(C\) y familia de aplicaciones \(\{\phi_i : C \rightarrow A_i | i \in I\}\), existe una única aplicación \(\psi : C \rightarrow D\) tal que \(\pi_i \phi = \phi_i\) para todo \(i \in I\).
    Más aún, \(D\) queda determinado univocamente salvo una biyección.
    \[
      \begin{tikzcd}
        C \arrow{rd}{\phi_i} \arrow[dotted]{d}{\phi} \\
        D \arrow{r}{\pi_i} & A_i
      \end{tikzcd}
    \]
  \end{theorem}
\end{document}