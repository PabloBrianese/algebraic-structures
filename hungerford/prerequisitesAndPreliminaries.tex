\documentclass{report}
\usepackage[spanish]{babel}
\title{Prerequisitos y preliminares \\ Hungerford}
\author{Pablo Brianese}

\usepackage{mathtools, amsthm, amssymb}
\usepackage{tikz-cd}

\newtheorem{theorem}{Teorema}
\begin{document}
  \maketitle

  \begin{theorem}[4.1]
    Sea \(A\) un conjunto novacio.
    Dada una relación de equivalencia \(R \subseteq A \times A\), definimos sus clases de equivalencia como \(\bar{a} = \{b \in A : (a, b) \in R\}\) para cada \(a \in A\), y definimos el cociente de \(A\) por \(R\) como \(A / R = \{\bar{a} : a \in A\}\).
    La asignación \(R \mapsto A / R\) define una biyección entre el conjunto \(E(A)\), formado por todas las relaciones de equivalencia \(R\) sobre \(A\), y el conjunto \(Q(A)\), formado por todas las parciciones de \(A\).
  \end{theorem}

\end{document}