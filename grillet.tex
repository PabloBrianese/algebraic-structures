\documentclass{article}

% Packages
\usepackage[a4paper, margin=1in]{geometry} % Adjusts the margins to 1 inch on all sides
\usepackage{amsmath} % For mathematical symbols and equations
\usepackage{amssymb} % For additional mathematical symbols
\usepackage{graphicx} % For including images
\usepackage{mathtools} % For mathematical environments
\usepackage{lipsum} % For generating dummy text
\usepackage[spanish]{babel} % For Spanish language support

\usepackage{amsthm} % For theorem environments

% Theorem environments
\newtheorem{definition}{Definición}[section]
\newtheorem{theorem}{Teorema}[section]
\newtheorem{corollary}{Corolario}[section]
\newtheorem{lemma}{Lema}[section]
\newtheorem{proposition}{Proposición}[section]

% Macro for endomorphisms
\newcommand{\End}{\text{End}}

% Macro for a ring R as an R module
\newcommand{\RAsRModule}{\prescript{}{R}{R}}

% Title and author
\title{El teorema de Artin-Wedderburn}
\author{Pablo Brianese en base a Grillet}

\begin{document}

\maketitle

% Page 370

El teorema de Artin--Wedderburn construye todos los anillos $R$, llamados semisimples, tales que todo $R$--módulo izquierdo es semisimple\footnote{Un $R$--módulo izquierdo es semisimple si es suma directa de simples. Uno es simple cuando es igual a su único submódulo no trivial}. El paso del tiempo lo ha mantenido como un resultado fundamental de la teoría de anillos.

Al igual que en el resto de este capítulo, todos los anillos tienen un elemento identidad, y todos los módulos son unitales. (El resultado principal se mantiene aún sin esta restricción; ver los ejercicios.)

\begin{definition}
    Un anillo $R$ es \emph{semisimple} cuando todo R--módulo izquierdo es semisimple.
\end{definition}

En rigor estos anillos deberían ser llamados \emph{anillos semisimples a izquierda}; pero mostraremos que $R$ es semisimple si y solo si todo $R$--módulo derecho es semisimple.

\end{document}